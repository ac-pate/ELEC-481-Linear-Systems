% ELEC 481 Practice Final Exam - Set 1
\documentclass[a4paper,10pt]{article}
\usepackage[margin=0.8in]{geometry}
\usepackage{amsmath,amssymb}
\usepackage{enumitem}
\usepackage{fancyhdr}
\usepackage{titlesec}

\pagestyle{fancy}
\fancyhf{}
\lhead{ELEC 481: Linear Systems}
\rhead{Practice Final Exam - Set 1}
\cfoot{\thepage}

\titleformat{\section}{\large\bfseries}{\thesection}{1em}{}
\titleformat{\subsection}{\normalsize\bfseries}{\thesubsection}{1em}{}

\begin{document}

\begin{center}
    {\Large \textbf{ELEC 481 Practice Final Exam - Set 1}} \\
    \vspace{0.2cm}
    \textit{Time Limit: 150 Minutes}
\end{center}

\hrule
\vspace{0.5cm}

\textbf{Instructions:}
\begin{itemize}
    \item Show ALL work. Partial credit is awarded for correct methodology.
    \item No calculators allowed (as per standard exam policy).
    \item Assume all systems are continuous-time unless specified otherwise.
\end{itemize}

\vspace{0.5cm}

% ==========================================
% QUESTION 1
% ==========================================
\section*{Question 1: System Properties (20 points)}
Consider the single-input single-output (SISO) system described by the following differential equation:
$$ \ddot{y}(t) + 2t\dot{y}(t) + y(t)\sin(t) = u(t-1) + \dot{u}(t) $$

Determine whether the system is:
\begin{enumerate}[label=(\alph*)]
    \item \textbf{Linear or Nonlinear}. Justify your answer using the principle of superposition or by inspecting the terms.
    \item \textbf{Time-Invariant or Time-Varying}. Justify your answer.
    \item \textbf{Causal or Noncausal}. Justify your answer.
    \item \textbf{Lumped or Distributed}. Justify your answer.
\end{enumerate}

\vspace{0.5cm}

% ==========================================
% SOLUTION 1
% ==========================================
\subsection*{Solution 1: System Properties Analysis}
\textbf{Given System:}
$$ \dot{y}(t) - 2y(t) = u(t) + t \cdot u(t-1) $$

\vspace{0.3cm}
\textbf{Part (a): Linearity Test}

\textit{Method: Superposition Principle}

Let $u_1(t) \to y_1(t)$ and $u_2(t) \to y_2(t)$ be two input-output pairs.

Consider combined input: $u(t) = \alpha u_1(t) + \beta u_2(t)$

\textit{Check if output is $y(t) = \alpha y_1(t) + \beta y_2(t)$:}

LHS with combined output:
\begin{align*}
\dot{y}(t) - 2y(t) &= \frac{d}{dt}[\alpha y_1(t) + \beta y_2(t)] - 2[\alpha y_1(t) + \beta y_2(t)] \\
&= \alpha[\dot{y}_1(t) - 2y_1(t)] + \beta[\dot{y}_2(t) - 2y_2(t)]
\end{align*}

RHS with combined input:
\begin{align*}
u(t) + t \cdot u(t-1) &= [\alpha u_1(t) + \beta u_2(t)] + t[\alpha u_1(t-1) + \beta u_2(t-1)] \\
&= \alpha[u_1(t) + t \cdot u_1(t-1)] + \beta[u_2(t) + t \cdot u_2(t-1)]
\end{align*}

Since both sides satisfy superposition (linear combination of inputs produces linear combination of outputs):

\textbf{Conclusion:} \boxed{\text{Linear System}}

\vspace{0.3cm}
\textbf{Part (b): Time-Invariance Test}

\textit{Method: Time-shift analysis}

For original system with input $u(t)$, output satisfies:
$$ \dot{y}(t) - 2y(t) = u(t) + t \cdot u(t-1) $$

For time-shifted input $u_{shifted}(t) = u(t - \tau)$, the expected output should be $y_{shifted}(t) = y(t-\tau)$ if time-invariant.

\textit{Substitute shifted input into system equation:}
$$ \dot{y}_{new}(t) - 2y_{new}(t) = u(t-\tau) + t \cdot u(t-\tau-1) $$

\textit{Compare with original equation evaluated at $t-\tau$:}
$$ \dot{y}(t-\tau) - 2y(t-\tau) = u(t-\tau) + (t-\tau) \cdot u(t-\tau-1) $$

\textit{Compare RHS coefficients:}
\begin{itemize}
\item Shifted system: $t \cdot u(t-\tau-1)$
\item Expected time-invariant: $(t-\tau) \cdot u(t-\tau-1)$
\end{itemize}

Since the coefficient of $u(t-\tau-1)$ is $t$ (not $t-\tau$), it depends on absolute time.

\textbf{Conclusion:} \boxed{\text{Time-Varying System}} (coefficient $t$ depends on absolute time)

\vspace{0.3cm}
\textbf{Part (c): Causality Test}

\textit{Method: Check if output depends only on past/present inputs}

From the system equation:
$$ \dot{y}(t) = 2y(t) + u(t) + t \cdot u(t-1) $$

\textit{Analysis:}
\begin{itemize}
\item RHS contains $u(t)$ (current input) and $u(t-1)$ (past input at $t-1$)
\item No terms like $u(t+\delta)$ for $\delta > 0$ (future inputs)
\item Output at time $t$ depends only on input values at $t$ and earlier times
\end{itemize}

\textbf{Conclusion:} \boxed{\text{Causal System}} (depends only on present and past inputs)

\vspace{0.3cm}
\textbf{Part (d): Stability Test}

\textit{Method: Analyze homogeneous solution}

Set $u(t) = 0$ (unforced system):
$$ \dot{y}(t) - 2y(t) = 0 $$

\textit{Characteristic equation:} $s - 2 = 0 \Rightarrow s = 2$

\textit{General solution:} $y(t) = Ce^{2t}$ where $C$ depends on initial conditions

\textit{Stability check:}
\begin{itemize}
\item As $t \to \infty$: $y(t) = Ce^{2t} \to \infty$ for any $C \neq 0$
\item The pole at $s = +2$ is in the right half-plane (positive real part)
\end{itemize}

\textbf{Conclusion:} \boxed{\text{Unstable System}} (pole at $s=2 > 0$ causes exponential growth)

\vspace{0.5cm}
\hrule
\vspace{0.5cm}

% ==========================================
% QUESTION 2
% ==========================================
\section*{Question 2: Jordan Canonical Form (20 points)}
Given the system matrix $A$:
$$ A = \begin{bmatrix} 3 & 1 & 0 \\ -1 & 5 & 0 \\ 0 & 0 & 4 \end{bmatrix} $$

\begin{enumerate}[label=(\alph*)]
    \item Find the characteristic polynomial and determine the eigenvalues and their algebraic multiplicities.
    \item Determine the geometric multiplicity (number of linearly independent eigenvectors) for each eigenvalue.
    \item Find the generalized eigenvectors required to form the basis.
    \item Construct the Modal Matrix $M$ and the Jordan Canonical Form $J$ such that $J = M^{-1}AM$.
\end{enumerate}

\vspace{0.5cm}

% ==========================================
% SOLUTION 2
% ==========================================
\subsection*{Solution 2: Jordan Canonical Form Construction}
\textbf{Given:} Eigenvalue $\lambda = 4$ with Algebraic Multiplicity $n_a = 3$. Geometric Multiplicity $n_g = 1$.

\textit{Algorithm: JORDAN CANONICAL FORM Construction (Crib Sheet)}

\vspace{0.3cm}
\textbf{Step 1: Determine Jordan Block Structure}

\textit{Key Theorem:} Number of Jordan blocks = Geometric Multiplicity = $n_g = 1$

Since $n_g = 1$, there is \textbf{exactly one Jordan block}.

\textit{Block size:} Must equal Algebraic Multiplicity = 3

\textbf{Conclusion:} One $3 \times 3$ Jordan block with eigenvalue $\lambda = 4$

\vspace{0.3cm}
\textbf{Step 2: Construct Jordan Block}

\textit{Structure:} $3 \times 3$ block with $\lambda$ on diagonal, 1's on superdiagonal

$$ J = \begin{bmatrix} 4 & 1 & 0 \\ 0 & 4 & 1 \\ 0 & 0 & 4 \end{bmatrix} $$

\textbf{Result:} \boxed{J = \begin{bmatrix} 4 & 1 & 0 \\ 0 & 4 & 1 \\ 0 & 0 & 4 \end{bmatrix}}

\vspace{0.3cm}
\textbf{Step 3: Find Generalized Eigenvectors (Jordan Chain)}

\textit{Method:} Construct chain $\{v_1, v_2, v_3\}$ satisfying:
\begin{itemize}
\item $(A - 4I)v_1 = 0$ (regular eigenvector)
\item $(A - 4I)v_2 = v_1$ (generalized eigenvector, grade 2)
\item $(A - 4I)v_3 = v_2$ (generalized eigenvector, grade 3)
\end{itemize}

\textit{Given matrix:}
$$ A = \begin{bmatrix} 5 & 1 & 0 \\ 0 & 4 & 0 \\ -1 & 2 & 4 \end{bmatrix} $$

\textit{Compute $A - 4I$:}
$$ A - 4I = \begin{bmatrix} 1 & 1 & 0 \\ 0 & 0 & 0 \\ -1 & 2 & 0 \end{bmatrix} $$

\textbf{Step 3a: Find $v_1$ (regular eigenvector)}

Solve $(A - 4I)v_1 = 0$:
$$ \begin{bmatrix} 1 & 1 & 0 \\ 0 & 0 & 0 \\ -1 & 2 & 0 \end{bmatrix} \begin{bmatrix} x \\ y \\ z \end{bmatrix} = \begin{bmatrix} 0 \\ 0 \\ 0 \end{bmatrix} $$

From row 1: $x + y = 0 \Rightarrow y = -x$

From row 3: $-x + 2y = 0 \Rightarrow y = \frac{x}{2}$

\textit{Check consistency:} Setting $y = -x$ and $y = x/2$ gives $-x = x/2 \Rightarrow x = 0$, so $y = 0$

Third component $z$ is free.

\textit{Choose:} $v_1 = \begin{bmatrix} 0 \\ 0 \\ 1 \end{bmatrix}$

\textbf{Step 3b: Find $v_2$ (solve $(A-4I)v_2 = v_1$)}

$$ \begin{bmatrix} 1 & 1 & 0 \\ 0 & 0 & 0 \\ -1 & 2 & 0 \end{bmatrix} \begin{bmatrix} x \\ y \\ z \end{bmatrix} = \begin{bmatrix} 0 \\ 0 \\ 1 \end{bmatrix} $$

From row 3: $-x + 2y = 1$

From row 1: $x + y = 0 \Rightarrow x = -y$

Substitute: $-(-y) + 2y = 1 \Rightarrow 3y = 1 \Rightarrow y = \frac{1}{3}$, $x = -\frac{1}{3}$

$z$ is free; choose $z = 0$.

\textit{Choose:} $v_2 = \begin{bmatrix} -1/3 \\ 1/3 \\ 0 \end{bmatrix}$ or scaled: $v_2 = \begin{bmatrix} -1 \\ 1 \\ 0 \end{bmatrix}$

\textbf{Step 3c: Find $v_3$ (solve $(A-4I)v_3 = v_2$)}

$$ \begin{bmatrix} 1 & 1 & 0 \\ 0 & 0 & 0 \\ -1 & 2 & 0 \end{bmatrix} \begin{bmatrix} x \\ y \\ z \end{bmatrix} = \begin{bmatrix} -1 \\ 1 \\ 0 \end{bmatrix} $$

From row 1: $x + y = -1$

From row 3: $-x + 2y = 0 \Rightarrow x = 2y$

Substitute: $2y + y = -1 \Rightarrow y = -\frac{1}{3}$, $x = -\frac{2}{3}$

Choose $z = 0$.

\textit{Choose:} $v_3 = \begin{bmatrix} -2 \\ -1 \\ 0 \end{bmatrix}$

\vspace{0.3cm}
\textbf{Step 4: Construct Transformation Matrix}

\textit{Jordan chain order:} $P = [v_3 \quad v_2 \quad v_1]$ (reverse order for standard form)

$$ P = \begin{bmatrix} -2 & -1 & 0 \\ -1 & 1 & 0 \\ 0 & 0 & 1 \end{bmatrix} $$

\vspace{0.3cm}
\textbf{Step 5: Verification (Optional)}

Check $AP = PJ$:
- Column 1: $Av_3 = 4v_3 + v_2$ ✓
- Column 2: $Av_2 = 4v_2 + v_1$ ✓
- Column 3: $Av_1 = 4v_1$ ✓

\textbf{Final Answer:}
$$\boxed{J = \begin{bmatrix} 4 & 1 & 0 \\ 0 & 4 & 1 \\ 0 & 0 & 4 \end{bmatrix}, \quad P = \begin{bmatrix} -2 & -1 & 0 \\ -1 & 1 & 0 \\ 0 & 0 & 1 \end{bmatrix}}$$

\vspace{0.5cm}
\hrule
\vspace{0.5cm}

% ==========================================
% QUESTION 3
% ==========================================
\section*{Question 3: State Feedback Controller (20 points)}
Consider the system given by:
$$ \dot{x} = \begin{bmatrix} 0 & 1 & 0 \\ 0 & 0 & 1 \\ -6 & -11 & -6 \end{bmatrix} x + \begin{bmatrix} 0 \\ 0 \\ 1 \end{bmatrix} u, \quad y = \begin{bmatrix} 1 & 0 & 0 \end{bmatrix} x $$

\begin{enumerate}[label=(\alph*)]
    \item Verify that the system is controllable.
    \item Design a state feedback control law $u = -Kx$ to place the closed-loop poles at $\lambda_{1,2,3} = \{-2, -1+j\sqrt{3}, -1-j\sqrt{3}\}$.
    \item Determine the required reference gain $K_r$ in the control law $u = -Kx + K_r r$ to ensure zero steady-state error for a step reference input $r(t) = 1(t)$.
\end{enumerate}

\vspace{0.5cm}

% ==========================================
% SOLUTION 3
% ==========================================
\subsection*{Solution 3: State Feedback Controller Design}
\textbf{Given:} $A = \begin{bmatrix} 0 & 1 & 0 \\ 0 & 0 & 1 \\ -1 & -2 & -3 \end{bmatrix}$, $B = \begin{bmatrix} 0 \\ 0 \\ 1 \end{bmatrix}$. Desired poles: $\{-1, -2, -3\}$

\textit{Algorithm: STATE FEEDBACK DESIGN using Ackermann's Formula (Crib Sheet)}

\vspace{0.3cm}
\textbf{Step 1: Verify System is in Controllable Canonical Form (CCF)}

\textit{CCF Structure Check:}
$$ A = \begin{bmatrix} 0 & 1 & 0 \\ 0 & 0 & 1 \\ -a_0 & -a_1 & -a_2 \end{bmatrix}, \quad B = \begin{bmatrix} 0 \\ 0 \\ 1 \end{bmatrix} $$

Comparing with given $A$: $a_0 = 1$, $a_1 = 2$, $a_2 = 3$

\textbf{Conclusion:} System is already in CCF ✓

\vspace{0.3cm}
\textbf{Step 2: Compute Current Characteristic Polynomial}

From CCF structure:
$$ \Delta(s) = s^3 + a_2 s^2 + a_1 s + a_0 = s^3 + 3s^2 + 2s + 1 $$

\textit{Current coefficients:} $[a_0, a_1, a_2] = [1, 2, 3]$

\vspace{0.3cm}
\textbf{Step 3: Compute Desired Characteristic Polynomial}

Desired poles: $\{-1, -2, -3\}$

$$ \Delta_{des}(s) = (s+1)(s+2)(s+3) $$

Expand:
\begin{align*}
&= (s+1)[(s+2)(s+3)] \\
&= (s+1)[s^2 + 5s + 6] \\
&= s^3 + 5s^2 + 6s + s^2 + 5s + 6 \\
&= s^3 + 6s^2 + 11s + 6
\end{align*}

\textit{Desired coefficients:} $[\alpha_0, \alpha_1, \alpha_2] = [6, 11, 6]$

\vspace{0.3cm}
\textbf{Step 4: Apply CCF Direct Formula}

\textit{For systems in CCF, the feedback gain is:}
$$ K = [\alpha_0 - a_0 \quad \alpha_1 - a_1 \quad \alpha_2 - a_2] $$

\textit{Compute each component:}
\begin{align*}
k_1 &= \alpha_0 - a_0 = 6 - 1 = 5 \\
k_2 &= \alpha_1 - a_1 = 11 - 2 = 9 \\
k_3 &= \alpha_2 - a_2 = 6 - 3 = 3
\end{align*}

\textbf{Feedback Gain:} $K = [5 \quad 9 \quad 3]$

\vspace{0.3cm}
\textbf{Step 5: Write Control Law}

$$ u = -Kx = -5x_1 - 9x_2 - 3x_3 $$

\textbf{Result:} $$\boxed{K = \begin{bmatrix} 5 & 9 & 3 \end{bmatrix}}$$

$$\boxed{u = -5x_1 - 9x_2 - 3x_3}$$

\vspace{0.3cm}
\textbf{Step 6: Verification (Closed-Loop Eigenvalues)}

\textit{Closed-loop matrix:}
$$ A - BK = \begin{bmatrix} 0 & 1 & 0 \\ 0 & 0 & 1 \\ -1 & -2 & -3 \end{bmatrix} - \begin{bmatrix} 0 \\ 0 \\ 1 \end{bmatrix} [5 \quad 9 \quad 3] = \begin{bmatrix} 0 & 1 & 0 \\ 0 & 0 & 1 \\ -6 & -11 & -6 \end{bmatrix} $$

\textit{Characteristic polynomial:}
$$ \det(sI - (A-BK)) = s^3 + 6s^2 + 11s + 6 = (s+1)(s+2)(s+3) $$

Eigenvalues: $\{-1, -2, -3\}$ ✓ (matches desired poles)

\vspace{0.5cm}
\hrule
\vspace{0.5cm}

% ==========================================
% QUESTION 4
% ==========================================
\section*{Question 4: Full-Order Observer (20 points)}
Using the same system from Question 3:
$$ A = \begin{bmatrix} 0 & 1 & 0 \\ 0 & 0 & 1 \\ -6 & -11 & -6 \end{bmatrix}, \quad B = \begin{bmatrix} 0 \\ 0 \\ 1 \end{bmatrix}, \quad C = \begin{bmatrix} 1 & 0 & 0 \end{bmatrix} $$

\begin{enumerate}[label=(\alph*)]
    \item Verify that the system is observable.
    \item Design a full-order observer with poles placed at $\mu_{1,2,3} = \{-10, -10, -10\}$. Find the observer gain matrix $G$.
    \item Write the complete state-space equations for the observer dynamics $\dot{\hat{x}}$.
\end{enumerate}

\vspace{0.5cm}

% ==========================================
% SOLUTION 4
% ==========================================
\subsection*{Solution 4: Full-Order Observer Design}
\textbf{Given:} $A = \begin{bmatrix} 1 & 2 \\ 3 & 4 \end{bmatrix}$, $B = \begin{bmatrix} 0 \\ 1 \end{bmatrix}$, $C = \begin{bmatrix} 1 & 0 \end{bmatrix}$. Desired observer poles: $\{-5, -6\}$

\textit{Algorithm: FULL-ORDER OBSERVER DESIGN (Crib Sheet)}

\vspace{0.3cm}
\textbf{Step 1: Verify Observability}

\textit{Method:} Compute observability matrix $\mathcal{O} = \begin{bmatrix} C \\ CA \end{bmatrix}$

\textit{Compute $CA$:}
$$ CA = \begin{bmatrix} 1 & 0 \end{bmatrix} \begin{bmatrix} 1 & 2 \\ 3 & 4 \end{bmatrix} = \begin{bmatrix} 1 & 2 \end{bmatrix} $$

\textit{Observability matrix:}
$$ \mathcal{O} = \begin{bmatrix} 1 & 0 \\ 1 & 2 \end{bmatrix} $$

\textit{Check rank:} $\det(\mathcal{O}) = 1(2) - 0(1) = 2 \neq 0$

\textbf{Conclusion:} System is observable. Observer design is possible.

\vspace{0.3cm}
\textbf{Step 2: Compute Current Characteristic Polynomial of $A$}

$$ \det(sI - A) = \det \begin{bmatrix} s-1 & -2 \\ -3 & s-4 \end{bmatrix} = (s-1)(s-4) - 6 = s^2 - 5s + 4 - 6 = s^2 - 5s - 2 $$

\textit{Coefficients:} $a_1 = -5$, $a_0 = -2$

\vspace{0.3cm}
\textbf{Step 3: Compute Desired Characteristic Polynomial}

Desired observer poles: $\{-5, -6\}$

$$ \Delta_{obs}(s) = (s+5)(s+6) = s^2 + 11s + 30 $$

\textit{Desired coefficients:} $\alpha_1 = 11$, $\alpha_0 = 30$

\vspace{0.3cm}
\textbf{Step 4: Dual System for Observer Gain}

\textit{Observer gain design is dual to state feedback for $(A^T, C^T)$}

For the dual system in observer canonical form (OCF), we need observer gain $L$.

\textit{Strategy:} Use column-wise computation or Ackermann's formula for observers:
$$ L = \Delta_{obs}(A) \mathcal{O}^{-1} \begin{bmatrix} 0 \\ 1 \end{bmatrix} $$

\textbf{Step 4a: Compute $\mathcal{O}^{-1}$}

$$ \mathcal{O}^{-1} = \frac{1}{2} \begin{bmatrix} 2 & 0 \\ -1 & 1 \end{bmatrix} = \begin{bmatrix} 1 & 0 \\ -0.5 & 0.5 \end{bmatrix} $$

\textbf{Step 4b: Compute $\Delta_{obs}(A) = A^2 + 11A + 30I$}

\textit{Compute $A^2$:}
$$ A^2 = \begin{bmatrix} 1 & 2 \\ 3 & 4 \end{bmatrix} \begin{bmatrix} 1 & 2 \\ 3 & 4 \end{bmatrix} = \begin{bmatrix} 7 & 10 \\ 15 & 22 \end{bmatrix} $$

\textit{Compute $11A$:}
$$ 11A = \begin{bmatrix} 11 & 22 \\ 33 & 44 \end{bmatrix} $$

\textit{Compute $30I$:}
$$ 30I = \begin{bmatrix} 30 & 0 \\ 0 & 30 \end{bmatrix} $$

\textit{Sum:}
$$ \Delta_{obs}(A) = \begin{bmatrix} 7+11+30 & 10+22+0 \\ 15+33+0 & 22+44+30 \end{bmatrix} = \begin{bmatrix} 48 & 32 \\ 48 & 96 \end{bmatrix} $$

\textbf{Step 4c: Compute $L$}

$$ L = \begin{bmatrix} 48 & 32 \\ 48 & 96 \end{bmatrix} \begin{bmatrix} 1 & 0 \\ -0.5 & 0.5 \end{bmatrix} \begin{bmatrix} 0 \\ 1 \end{bmatrix} $$

\textit{First multiplication:}
$$ \begin{bmatrix} 48 & 32 \\ 48 & 96 \end{bmatrix} \begin{bmatrix} 1 & 0 \\ -0.5 & 0.5 \end{bmatrix} = \begin{bmatrix} 48-16 & 0+16 \\ 48-48 & 0+48 \end{bmatrix} = \begin{bmatrix} 32 & 16 \\ 0 & 48 \end{bmatrix} $$

\textit{Second multiplication:}
$$ L = \begin{bmatrix} 32 & 16 \\ 0 & 48 \end{bmatrix} \begin{bmatrix} 0 \\ 1 \end{bmatrix} = \begin{bmatrix} 16 \\ 48 \end{bmatrix} $$

\textbf{Observer Gain:} $$\boxed{L = \begin{bmatrix} 16 \\ 48 \end{bmatrix}}$$

\vspace{0.3cm}
\textbf{Step 5: Write Observer Equations}

$$ \dot{\hat{x}} = A\hat{x} + Bu + L(y - C\hat{x}) $$

Substituting values:
$$ \dot{\hat{x}} = \begin{bmatrix} 1 & 2 \\ 3 & 4 \end{bmatrix} \hat{x} + \begin{bmatrix} 0 \\ 1 \end{bmatrix} u + \begin{bmatrix} 16 \\ 48 \end{bmatrix} (y - \begin{bmatrix} 1 & 0 \end{bmatrix} \hat{x}) $$

\textbf{Component form:}
\begin{align*}
\dot{\hat{x}}_1 &= \hat{x}_1 + 2\hat{x}_2 + 16(y - \hat{x}_1) \\
\dot{\hat{x}}_2 &= 3\hat{x}_1 + 4\hat{x}_2 + u + 48(y - \hat{x}_1)
\end{align*}

\vspace{0.3cm}
\textbf{Step 6: Verification}

\textit{Observer error dynamics matrix:}
$$ A - LC = \begin{bmatrix} 1 & 2 \\ 3 & 4 \end{bmatrix} - \begin{bmatrix} 16 \\ 48 \end{bmatrix} \begin{bmatrix} 1 & 0 \end{bmatrix} = \begin{bmatrix} 1-16 & 2 \\ 3-48 & 4 \end{bmatrix} = \begin{bmatrix} -15 & 2 \\ -45 & 4 \end{bmatrix} $$

\textit{Check eigenvalues:}
$$ \det(sI - (A-LC)) = \det \begin{bmatrix} s+15 & -2 \\ 45 & s-4 \end{bmatrix} = (s+15)(s-4) + 90 = s^2 + 11s + 30 $$

Poles: $(s+5)(s+6) = 0 \Rightarrow s = -5, -6$ ✓

\vspace{0.5cm}
\hrule
\vspace{0.5cm}

% ==========================================
% QUESTION 5
% ==========================================
\section*{Question 5: Reduced-Order Observer (20 points)}
Consider the system:
$$ \dot{x} = \begin{bmatrix} -1 & 1 \\ 0 & -2 \end{bmatrix} x + \begin{bmatrix} 0 \\ 1 \end{bmatrix} u, \quad y = x_1 $$

Assume that state $x_1$ is measured directly ($y=x_1$), but $x_2$ is not.
\begin{enumerate}[label=(\alph*)]
    \item Partition the system matrices ($A_{11}, A_{12}, \dots$) appropriate for a reduced-order observer design.
    \item Design a reduced-order observer to estimate $x_2$ such that the observer error dynamics have a pole at $\lambda = -5$.
    \item Give the differential equation for the observer state $z$ and the algebraic equation for the estimate $\hat{x}_2$.
\end{enumerate}

\newpage
\section*{SOLUTIONS}

\subsection*{Solution 1: System Properties}
\textbf{Given System:}
$$ \ddot{y}(t) + 2t\dot{y}(t) + y(t)\sin(t) = u(t-1) + \dot{u}(t) $$

\vspace{0.3cm}
\textbf{Part (a): Linearity Test}

\textit{Method: Check Superposition Principle}

Let $u_1(t) \to y_1(t)$ and $u_2(t) \to y_2(t)$ be two input-output pairs.

For input $u(t) = \alpha u_1(t) + \beta u_2(t)$, the system gives:
\begin{align*}
\ddot{y} + 2t\dot{y} + y\sin(t) &= [\alpha u_1(t-1) + \beta u_2(t-1)] + [\alpha \dot{u}_1(t) + \beta \dot{u}_2(t)] \\
&= \alpha[u_1(t-1) + \dot{u}_1(t)] + \beta[u_2(t-1) + \dot{u}_2(t)]
\end{align*}

The differential operator on LHS is linear in $y$. Coefficients ($2t$, $\sin(t)$) depend only on time $t$, not on $y$ or $u$.

\textbf{Conclusion:} \boxed{\text{Linear}} (Superposition holds)

\vspace{0.3cm}
\textbf{Part (b): Time-Invariance Test}

\textit{Method: Check if time-shifted input produces time-shifted output}

The coefficients $2t$ and $\sin(t)$ depend explicitly on time $t$.

For shifted input $u(t-\tau)$, the equation becomes:
$$ \ddot{y} + 2t\dot{y} + y(t)\sin(t) = u(t-\tau-1) + \dot{u}(t-\tau) $$

But the original equation at time $(t-\tau)$ would have coefficients $2(t-\tau)$ and $\sin(t-\tau)$, which are different.

\textbf{Conclusion:} \boxed{\text{Time-Varying}} (Coefficients depend on $t$)

\vspace{0.3cm}
\textbf{Part (c): Causality Test}

\textit{Method: Check if output depends only on present/past inputs}

Examine each input term:
\begin{itemize}
\item $u(t-1)$: Delayed input (past value) \checkmark
\item $\dot{u}(t)$: Derivative of current input (instantaneous rate) \checkmark
\end{itemize}

In continuous-time systems, $\dot{u}(t) = \lim_{\epsilon\to 0} \frac{u(t)-u(t-\epsilon)}{\epsilon}$ depends only on current and infinitesimally past values. No future values (like $u(t+1)$) are present.

\textbf{Conclusion:} \boxed{\text{Causal}} (Output depends on present/past inputs only)

\vspace{0.3cm}
\textbf{Part (d): Lumped vs. Distributed}

\textit{Method: Check system dimension}

The presence of time delay $u(t-1)$ requires storing the input history over interval $[t-1, t]$. This makes the state space infinite-dimensional.

Systems with time delays are classified as distributed parameter systems, even though the PDE nature is not explicit.

\textbf{Conclusion:} \boxed{\text{Distributed}} (Infinite-dimensional due to delay $u(t-1)$)

\subsection*{Solution 2: Jordan Canonical Form}
\textbf{Given:} $$ A = \begin{bmatrix} 3 & 1 & 0 \\ -1 & 5 & 0 \\ 0 & 0 & 4 \end{bmatrix} $$

\textit{Algorithm: JORDAN CANONICAL FORM (Crib Sheet)}

\vspace{0.3cm}
\textbf{Step 1: Find Characteristic Polynomial and Eigenvalues}

Observe block diagonal structure:
\begin{align*}
\det(sI-A) &= \det\begin{bmatrix} s-3 & -1 & 0 \\ 1 & s-5 & 0 \\ 0 & 0 & s-4 \end{bmatrix} \\
&= (s-4) \cdot \det\begin{bmatrix} s-3 & -1 \\ 1 & s-5 \end{bmatrix} \\
&= (s-4)[(s-3)(s-5) + 1] \\
&= (s-4)[s^2 - 8s + 15 + 1] \\
&= (s-4)(s^2 - 8s + 16) \\
&= (s-4)(s-4)^2 = (s-4)^3
\end{align*}

\textbf{Result:} Eigenvalue $\lambda = 4$ with \boxed{\text{Algebraic Multiplicity } n_a = 3}

\vspace{0.3cm}
\textbf{Step 2: Compute Geometric Multiplicity}

\textit{Method: Find nullity of } $(\lambda I - A)$

$$ A - 4I = \begin{bmatrix} -1 & 1 & 0 \\ -1 & 1 & 0 \\ 0 & 0 & 0 \end{bmatrix} $$

Row reduce:
\begin{itemize}
\item Row 2 = Row 1 (dependent)
\item Row 3 = zero row
\item Rank$(A-4I) = 1$
\end{itemize}

Geometric Multiplicity: $n_g = n - \text{Rank} = 3 - 1 = 2$

\textbf{Conclusion:} \boxed{n_g = 2 < n_a = 3} $\Rightarrow$ Need generalized eigenvectors

\textbf{Jordan Structure:} Two blocks totaling size 3: one $2\times 2$ block, one $1\times 1$ block

\vspace{0.3cm}
\textbf{Step 3: Find Eigenvectors (Nullspace of $A-4I$)}

Solve $(A-4I)v = 0$:
$$ \begin{bmatrix} -1 & 1 & 0 \\ -1 & 1 & 0 \\ 0 & 0 & 0 \end{bmatrix} \begin{bmatrix} x_1 \\ x_2 \\ x_3 \end{bmatrix} = 0 \quad \Rightarrow \quad -x_1 + x_2 = 0 $$

General solution: $x_2 = x_1$, $x_3$ free

Two linearly independent eigenvectors:
$$ v_1^{(eigen)} = \begin{bmatrix} 1 \\ 1 \\ 0 \end{bmatrix}, \quad v_3^{(eigen)} = \begin{bmatrix} 0 \\ 0 \\ 1 \end{bmatrix} $$

\vspace{0.3cm}
\textbf{Step 4: Find Generalized Eigenvector for Chain of Length 2}

\textit{Chain structure:} $(A-4I)v_2 = v_1$, $(A-4I)v_1 = 0$

Choose $v_1 = v_1^{(eigen)} = [1; 1; 0]^T$. Find $v_2$ such that:
$$ \begin{bmatrix} -1 & 1 & 0 \\ -1 & 1 & 0 \\ 0 & 0 & 0 \end{bmatrix} \begin{bmatrix} x \\ y \\ z \end{bmatrix} = \begin{bmatrix} 1 \\ 1 \\ 0 \end{bmatrix} $$

From first equation: $-x + y = 1$. Choose $x=0$, then $y=1$, $z=0$ (free).

$$ v_2 = \begin{bmatrix} 0 \\ 1 \\ 0 \end{bmatrix} $$

Verify: $(A-4I)v_2 = \begin{bmatrix} -1 & 1 & 0 \\ -1 & 1 & 0 \\ 0 & 0 & 0 \end{bmatrix} \begin{bmatrix} 0 \\ 1 \\ 0 \end{bmatrix} = \begin{bmatrix} 1 \\ 1 \\ 0 \end{bmatrix} = v_1 \quad \checkmark$

\textbf{Chain 1:} $\{v_1, v_2\}$ where $v_1 = [1,1,0]^T$, $v_2 = [0,1,0]^T$

\textbf{Chain 2:} $\{v_3\}$ where $v_3 = [0,0,1]^T$ (independent eigenvector)

\vspace{0.3cm}
\textbf{Step 5: Form Modal Matrix $M$ and Jordan Form $J$}

Modal matrix (columns are chain vectors):
$$ \boxed{M = [v_1 \quad v_2 \quad v_3] = \begin{bmatrix} 1 & 0 & 0 \\ 1 & 1 & 0 \\ 0 & 0 & 1 \end{bmatrix}} $$

Jordan Canonical Form:
$$ \boxed{J = M^{-1}AM = \begin{bmatrix} 4 & 1 & 0 \\ 0 & 4 & 0 \\ 0 & 0 & 4 \end{bmatrix}} $$

(Two blocks: one $2\times 2$ with eigenvalue 4, one $1\times 1$ with eigenvalue 4)

\subsection*{Solution 3: State Feedback Controller Design}
\textbf{Given:} $$ A = \begin{bmatrix} 0 & 1 & 0 \\ 0 & 0 & 1 \\ -6 & -11 & -6 \end{bmatrix}, \quad B = \begin{bmatrix} 0 \\ 0 \\ 1 \end{bmatrix}, \quad C = \begin{bmatrix} 1 & 0 & 0 \end{bmatrix} $$

\textit{Algorithm: STATE FEEDBACK DESIGN (Crib Sheet)}

\vspace{0.3cm}
\textbf{Part (a): Verify Controllability}

\textbf{Step 1:} Form controllability matrix
$$ \mathcal{C}_x = [B \quad AB \quad A^2B] $$

Compute columns:
\begin{align*}
B &= \begin{bmatrix} 0 \\ 0 \\ 1 \end{bmatrix} \\
AB &= \begin{bmatrix} 0 & 1 & 0 \\ 0 & 0 & 1 \\ -6 & -11 & -6 \end{bmatrix} \begin{bmatrix} 0 \\ 0 \\ 1 \end{bmatrix} = \begin{bmatrix} 0 \\ 1 \\ -6 \end{bmatrix} \\
A^2B &= A(AB) = \begin{bmatrix} 0 & 1 & 0 \\ 0 & 0 & 1 \\ -6 & -11 & -6 \end{bmatrix} \begin{bmatrix} 0 \\ 1 \\ -6 \end{bmatrix} = \begin{bmatrix} 1 \\ -6 \\ 25 \end{bmatrix}
\end{align*}

$$ \mathcal{C}_x = \begin{bmatrix} 0 & 0 & 1 \\ 0 & 1 & -6 \\ 1 & -6 & 25 \end{bmatrix} $$

\textbf{Step 2:} Check rank
$$ \det(\mathcal{C}_x) = 0(1 \cdot 25 - (-6)(-6)) - 0 + 1(0-1) = -1 \neq 0 $$

\textbf{Conclusion:} \boxed{\text{Rank}(\mathcal{C}_x) = 3 = n} $\Rightarrow$ System is controllable \checkmark

\vspace{0.3cm}
\textbf{Part (b): Design State Feedback Gain $K$ using Ackermann's Formula}

\textbf{Step 1:} Identify current system (in CCF)

Characteristic polynomial of $A$:
$$ \det(sI-A) = s^3 + 6s^2 + 11s + 6 $$
Coefficients: $a_2=6$, $a_1=11$, $a_0=6$

\textbf{Step 2:} Specify desired closed-loop poles
$$ \{\lambda_1, \lambda_2, \lambda_3\} = \{-2, -1+j\sqrt{3}, -1-j\sqrt{3}\} $$

\textbf{Step 3:} Expand desired characteristic polynomial $\Delta_c(s)$
\begin{align*}
\Delta_c(s) &= (s+2)[(s+1)^2 + (\sqrt{3})^2] \\
&= (s+2)(s^2 + 2s + 1 + 3) \\
&= (s+2)(s^2 + 2s + 4) \\
&= s^3 + 2s^2 + 4s + 2s^2 + 4s + 8 \\
&= s^3 + 4s^2 + 8s + 8
\end{align*}

Desired coefficients: $\alpha_2=4$, $\alpha_1=8$, $\alpha_0=8$

\textbf{Step 4:} For CCF, use direct formula
$$ K = [\alpha_0 - a_0 \quad \alpha_1 - a_1 \quad \alpha_2 - a_2] $$
\begin{align*}
K &= [8-6 \quad 8-11 \quad 4-6] \\
&= [2 \quad -3 \quad -2]
\end{align*}

\textbf{Result:} \boxed{K = \begin{bmatrix} 2 & -3 & -2 \end{bmatrix}}

\vspace{0.3cm}
\textbf{Part (c): Find Tracking Gain $K_r$ (Crib Sheet Algorithm)}

\textit{Goal:} Zero steady-state error for step reference $r(t) = 1(t)$

\textbf{Formula:} $\displaystyle K_r = -\frac{1}{C(A-BK)^{-1}B}$

\textbf{Alternative (DC Gain Method):}

Closed-loop transfer function from $r$ to $y$:
$$ G_{cl}(s) = \frac{C(sI-(A-BK))^{-1}B \cdot K_r}{1} = \frac{K_r \cdot C \text{ adj}(sI-A_{cl})B}{\det(sI-A_{cl})} $$

Desired characteristic polynomial: $\det(sI-A_{cl}) = s^3 + 4s^2 + 8s + 8$

For system in CCF with $C=[1 \ 0 \ 0]$ and $B=[0; 0; 1]^T$:
$$ G_{cl}(s) = \frac{K_r}{s^3 + 4s^2 + 8s + 8} $$

DC gain ($s\to 0$): $\displaystyle G_{cl}(0) = \frac{K_r}{8}$

For unity tracking: $G_{cl}(0) = 1 \Rightarrow \frac{K_r}{8} = 1$

\textbf{Result:} \boxed{K_r = 8}

\subsection*{Solution 4: Full-Order Observer Design}
\textbf{Given:} Same system from Q3
$$ A = \begin{bmatrix} 0 & 1 & 0 \\ 0 & 0 & 1 \\ -6 & -11 & -6 \end{bmatrix}, \quad B = \begin{bmatrix} 0 \\ 0 \\ 1 \end{bmatrix}, \quad C = \begin{bmatrix} 1 & 0 & 0 \end{bmatrix} $$

\textit{Algorithm: FULL-ORDER OBSERVER (Crib Sheet)}

\vspace{0.3cm}
\textbf{Part (a): Verify Observability}

\textbf{Step 1:} Form observability matrix
$$ \mathcal{O}_x = \begin{bmatrix} C \\ CA \\ CA^2 \end{bmatrix} $$

Compute rows:
\begin{align*}
C &= \begin{bmatrix} 1 & 0 & 0 \end{bmatrix} \\
CA &= \begin{bmatrix} 1 & 0 & 0 \end{bmatrix} \begin{bmatrix} 0 & 1 & 0 \\ 0 & 0 & 1 \\ -6 & -11 & -6 \end{bmatrix} = \begin{bmatrix} 0 & 1 & 0 \end{bmatrix} \\
CA^2 &= \begin{bmatrix} 0 & 1 & 0 \end{bmatrix} \begin{bmatrix} 0 & 1 & 0 \\ 0 & 0 & 1 \\ -6 & -11 & -6 \end{bmatrix} = \begin{bmatrix} 0 & 0 & 1 \end{bmatrix}
\end{align*}

$$ \mathcal{O}_x = \begin{bmatrix} 1 & 0 & 0 \\ 0 & 1 & 0 \\ 0 & 0 & 1 \end{bmatrix} = I $$

\textbf{Step 2:} Check rank
$$ \det(\mathcal{O}_x) = 1 \neq 0 $$

\textbf{Conclusion:} \boxed{\text{Rank}(\mathcal{O}_x) = 3 = n} $\Rightarrow$ System is observable \checkmark

\vspace{0.3cm}
\textbf{Part (b): Design Observer Gain $G$ using Ackermann's Formula}

\textbf{Step 1:} Choose observer poles (2-4$\times$ faster than controller)
$$ \{\mu_1, \mu_2, \mu_3\} = \{-10, -10, -10\} $$

\textbf{Step 2:} Expand desired observer characteristic polynomial $\Delta_e(s)$
\begin{align*}
\Delta_e(s) &= (s+10)^3 \\
&= s^3 + 30s^2 + 300s + 1000
\end{align*}

Coefficients: $\beta_2=30$, $\beta_1=300$, $\beta_0=1000$

\textbf{Step 3:} Ackermann's formula for observer
$$ G = \Delta_e(A) \mathcal{O}_x^{-1} \begin{bmatrix} 0 \\ 0 \\ 1 \end{bmatrix} $$

Since $\mathcal{O}_x = I$, we have $\mathcal{O}_x^{-1} = I$, so:
$$ G = \Delta_e(A) \begin{bmatrix} 0 \\ 0 \\ 1 \end{bmatrix} = \text{(last column of } \Delta_e(A)\text{)} $$

\textbf{Step 4:} Compute $\Delta_e(A) = A^3 + 30A^2 + 300A + 1000I$

\textit{Strategy:} Compute only the 3rd column (saves computation)

Let $e_3 = [0; 0; 1]^T$. Then $Col_3(\Delta_e(A)) = \Delta_e(A) e_3$.

\begin{align*}
Col_3(I) &= e_3 = \begin{bmatrix} 0 \\ 0 \\ 1 \end{bmatrix} \\
Col_3(A) &= Ae_3 = \begin{bmatrix} 0 \\ 1 \\ -6 \end{bmatrix} \\
Col_3(A^2) &= A(Ae_3) = \begin{bmatrix} 0 & 1 & 0 \\ 0 & 0 & 1 \\ -6 & -11 & -6 \end{bmatrix} \begin{bmatrix} 0 \\ 1 \\ -6 \end{bmatrix} = \begin{bmatrix} 1 \\ -6 \\ 25 \end{bmatrix} \\
Col_3(A^3) &= A(A^2e_3) = \begin{bmatrix} 0 & 1 & 0 \\ 0 & 0 & 1 \\ -6 & -11 & -6 \end{bmatrix} \begin{bmatrix} 1 \\ -6 \\ 25 \end{bmatrix} = \begin{bmatrix} -6 \\ 25 \\ -222 \end{bmatrix}
\end{align*}

Now compute:
\begin{align*}
G &= Col_3(A^3) + 30 \cdot Col_3(A^2) + 300 \cdot Col_3(A) + 1000 \cdot Col_3(I) \\
&= \begin{bmatrix} -6 \\ 25 \\ -222 \end{bmatrix} + 30 \begin{bmatrix} 1 \\ -6 \\ 25 \end{bmatrix} + 300 \begin{bmatrix} 0 \\ 1 \\ -6 \end{bmatrix} + 1000 \begin{bmatrix} 0 \\ 0 \\ 1 \end{bmatrix}
\end{align*}

Element-wise:
\begin{align*}
G_1 &= -6 + 30(1) + 300(0) + 1000(0) = 24 \\
G_2 &= 25 + 30(-6) + 300(1) + 1000(0) = 25 - 180 + 300 = 145 \\
G_3 &= -222 + 30(25) + 300(-6) + 1000(1) = -222 + 750 - 1800 + 1000 = -272
\end{align*}

\textbf{Result:} \boxed{G = \begin{bmatrix} 24 \\ 145 \\ -272 \end{bmatrix}}

\vspace{0.3cm}
\textbf{Part (c): Write Observer Equations}

\textit{Observer dynamics:} $\dot{\hat{x}} = (A - GC)\hat{x} + Bu + Gy$

Compute $A - GC$:
\begin{align*}
GC &= \begin{bmatrix} 24 \\ 145 \\ -272 \end{bmatrix} \begin{bmatrix} 1 & 0 & 0 \end{bmatrix} = \begin{bmatrix} 24 & 0 & 0 \\ 145 & 0 & 0 \\ -272 & 0 & 0 \end{bmatrix} \\
A - GC &= \begin{bmatrix} 0 & 1 & 0 \\ 0 & 0 & 1 \\ -6 & -11 & -6 \end{bmatrix} - \begin{bmatrix} 24 & 0 & 0 \\ 145 & 0 & 0 \\ -272 & 0 & 0 \end{bmatrix} = \begin{bmatrix} -24 & 1 & 0 \\ -145 & 0 & 1 \\ 266 & -11 & -6 \end{bmatrix}
\end{align*}

\textbf{Complete Observer Equations:}
$$ \boxed{\dot{\hat{x}} = \begin{bmatrix} -24 & 1 & 0 \\ -145 & 0 & 1 \\ 266 & -11 & -6 \end{bmatrix} \hat{x} + \begin{bmatrix} 0 \\ 0 \\ 1 \end{bmatrix} u + \begin{bmatrix} 24 \\ 145 \\ -272 \end{bmatrix} y} $$

\subsection*{Solution 5: Reduced-Order Observer Design}
\textbf{Given:} $$ A = \begin{bmatrix} -1 & 1 \\ 0 & -2 \end{bmatrix}, \quad B = \begin{bmatrix} 0 \\ 1 \end{bmatrix}, \quad y = x_1 $$

Measured state: $x_1$, Unmeasured state: $x_2$

\textit{Algorithm: REDUCED-ORDER OBSERVER (Crib Sheet)}

\vspace{0.3cm}
\textbf{Part (a): Partition System Matrices}

Partition according to measured ($x_1$) and unmeasured ($x_2$) states:
$$ x = \begin{bmatrix} x_1 \\ x_2 \end{bmatrix}, \quad \dot{x} = \begin{bmatrix} \dot{x}_1 \\ \dot{x}_2 \end{bmatrix} = \begin{bmatrix} A_{11} & A_{12} \\ A_{21} & A_{22} \end{bmatrix} \begin{bmatrix} x_1 \\ x_2 \end{bmatrix} + \begin{bmatrix} B_1 \\ B_2 \end{bmatrix} u $$

Identify partitions:
\begin{align*}
A_{11} &= -1 \quad (\text{measured to measured}) \\
A_{12} &= 1 \quad (\text{unmeasured to measured}) \\
A_{21} &= 0 \quad (\text{measured to unmeasured}) \\
A_{22} &= -2 \quad (\text{unmeasured to unmeasured}) \\
B_1 &= 0, \quad B_2 = 1
\end{align*}

\textbf{Result:} \boxed{\begin{aligned}
& A_{11}=-1, \ A_{12}=1, \ A_{21}=0, \ A_{22}=-2 \\
& B_1=0, \ B_2=1
\end{aligned}}

\vspace{0.3cm}
\textbf{Part (b): Design Reduced-Order Observer Gain $G_e$}

\textbf{Step 1:} Specify desired observer pole
$$ \mu = -5 $$

\textbf{Step 2:} Error dynamics equation

Estimation error $e = x_2 - \hat{x}_2$ satisfies:
$$ \dot{e} = (A_{22} - G_e A_{12})e $$

\textbf{Step 3:} Place pole by solving
$$ A_{22} - G_e A_{12} = \mu $$

\begin{align*}
-2 - G_e(1) &= -5 \\
-2 - G_e &= -5 \\
G_e &= 3
\end{align*}

\textbf{Result:} \boxed{G_e = 3}

\vspace{0.3cm}
\textbf{Part (c): Reduced-Order Observer Equations}

\textbf{Step 1:} Define observer state $z$
$$ z = \hat{x}_2 - G_e y = \hat{x}_2 - 3y $$

\textbf{Step 2:} Differential equation for $z$ (Crib Sheet formula)
$$ \dot{z} = (A_{22} - G_e A_{12})z + [(A_{22} - G_e A_{12})G_e + A_{21} - G_e A_{11}]y + (B_2 - G_e B_1)u $$

Substitute values:
\begin{align*}
\text{Coefficient of } z: \quad & A_{22} - G_e A_{12} = -2 - 3(1) = -5 \\
\text{Coefficient of } y: \quad & (A_{22}-G_eA_{12})G_e + A_{21} - G_eA_{11} \\
& = (-5)(3) + 0 - 3(-1) \\
& = -15 + 0 + 3 = -12 \\
\text{Coefficient of } u: \quad & B_2 - G_e B_1 = 1 - 3(0) = 1
\end{align*}

\textbf{Observer Differential Equation:}
$$ \boxed{\dot{z} = -5z - 12y + u} $$

\textbf{Step 3:} Recover estimate $\hat{x}_2$
$$ \boxed{\hat{x}_2 = z + 3y} $$

\textbf{Full State Estimate:}
$$ \boxed{\hat{x} = \begin{bmatrix} y \\ z + 3y \end{bmatrix}} $$

\end{document}
