%==============================================================================
% ELEC 481 Lab Report
%
% ==============================================================================

% hey guys, this is the main template for our report. i set up all the sections
% and packages so we should be good to go. just fill in the TODO parts.

% --- HOW TO USE THIS TEMPLATE (just fyi) ---
%
% TO ADD A FIGURE:
% 1. upload the image file to overleaf (the button on the top left).
% 2. uncomment the code block below and change the filename.
% 3. the label is super important for referencing it in the text, dont forget it.
%    (to reference it, you type:  blabla...as shown in Figure~\ref{fig:my_label} )
%
% TO ADD AN EQUATION:
% - use the 'equation' environment for one line, or the 'align' enviroment for multiple lines.
% - the '&' symbol is for lining them up (usually at the = sign).
% - the '\\' command makes a new line.
% - again, label it so we can reference it with \ref{...}.
%
% TABLES:
% - honestly the easiest way is to use a website like tablesgenerator.com and just paste the code here.
%   i put a placeholder for our parameters table below.
%
% CITING:
% - to cite something, just use \cite{label}. i already added the main books to the references section at the bottom. we can add more if we need to.
%
% COMMENTS:
% - use a '%' to write a comment, latex will ignore it.
%
% PDF VIEW:
% - To see the pdf, just hit the green 'Recompile' button.


\documentclass[12pt, letterpaper]{IEEEtran}

% --- PACKAGES ---
\usepackage[utf8]{inputenc}
\usepackage{amsmath}         % for math
\usepackage{graphicx}        % for images
\usepackage{booktabs}        % for nice tables
\usepackage{hyperref}        % for clickable links
\usepackage{float}             % for better figure placement

\usepackage{amsmath,amssymb}

\usepackage{siunitx}         % for SI units
\setlength{\headheight}{14.5pt}

% --- PAGE LAYOUT ---


\hypersetup{
    colorlinks=true, linkcolor=blue, urlcolor=cyan,
}


% --- TITLE PAGE INFO --- %


\begin{document}
\begin{titlepage}
\centering
\includegraphics[width=0.55\textwidth]{Other_figures/CU_logo.JPG}\\[2cm]

{\Large \textbf{Concordia University}}\\[0.5em]
{\large Gina Cody School of Engineering and Computer Science}\\[3cm]

{\LARGE \textbf{ELEC 481 - Final Project Report}}\\[0.5em]
{\Large  2-DOF Torsional Control System}\\[3cm]
% {\large Analysis of a 2-DOF Torsional Plant}\\[3cm]

{\large
Mathias Desrochers $|$ 40241734 $|$ eltopchi1@gmail.com\\[0.3em]
Achal Patel $|$ 40227663 $|$ achalypatel3403@gmail.com\\[0.3em]
Kenza Tarek $|$ 40224665 $|$ kenztar98@gmail.com\\ [3.0cm]}

{\large
Instructor: Dr.\ K.~Khorasani\\[0.3em]
Fall 2025
}\\[2.0cm]

{\large December 15, 2025}

\vfill
\end{titlepage}



% ==============================================================================
% --- DOCUMENT STARTS HERE ---
% ==============================================================================
\twocolumn[
\begin{abstract}

This report details the control of a two-degree-of-freedom torsional system using classical PID and modern state-feedback techniques. The fourth-order plant exhibits marginal stability and lightly damped resonant modes ($\zeta \approx 0.01$), posing significant control challenges. Design targets are 5\% overshoot and 2-second settling time.

A PID controller stabilizes the system but achieves only $\sim$20\% overshoot and 15--20s settling time due to insufficient degrees of freedom for independent pole placement. In contrast, state-feedback control with a Luenberger observer meets the specifications ($\sim$7\% overshoot, 1.9s settling time) by explicitly placing all closed-loop poles. Robustness testing shows that the observer-based design rejects measurement noise effectively, whereas the PID derivative term amplifies high-frequency disturbances. Results are validated through MATLAB/Simulink simulation, as hardware sensor malfunctions prevented experimental closed-loop testing on the physical apparatus..

\end{abstract}
\vspace{1em}
]
% ==============================================================================
% --- REPORT SECTIONS ---
% ==============================================================================

\section{Introduction}
% TODO: State the clear objective of the lab.
% Briefly describe the ECP Model 205 system.
% Outline the structure of the report (i.e., what each section covers).

Flexible mechanical systems with lightly damped resonant modes are commonly encountered in aerospace, robotics, and precision motion applications. Control of such systems presents significant challenges, as high performance requirements often conflict with stability and robustness constraints. The Control Moment Gyroscope (CMG) experimental platform provides a representative example of this class of systems and serves as a practical testbed for comparing classical and modern control strategies.

The objectives of this laboratory are threefold. First, to develop a mathematical model of a coupled torsional system starting from a transfer-function description and to analyze its open-loop dynamic properties. Second, to design and evaluate a classical PID controller using root locus and frequency-response techniques. Third, to implement a state-feedback controller with an observer and assess its ability to overcome the limitations observed with classical control.


\begin{figure}[H]
\centering
\includegraphics[width=0.85\columnwidth]{Other_figures/torscfg.png}
\caption{Torsional configurations and equivalent models considered in the modeling.}
\label{fig:tors_configs}
\end{figure}

\begin{figure}[H]
\centering
\includegraphics[width=0.5\columnwidth]{Other_figures/tors.png}
\caption{ECP torsional apparatus used in this laboratory.}
\label{fig:tors_apparatus}
\end{figure}

The laboratory emphasizes the relationship between system order, resonant dynamics, and achievable closed-loop performance. Particular attention is given to controllability, observability, pole placement, and noise robustness. By directly comparing PID and state-feedback control on the same plant, the lab highlights the practical trade-offs between implementation simplicity and performance guarantees in real-world control design.



Flexible mechanical systems with lightly damped resonant modes are commonly encountered in aerospace, robotics, and precision motion applications. Control of such systems presents significant challenges, as high performance requirements often conflict with stability and robustness constraints. The ECP Model 205 torsional system used in this laboratory provides a representative example of this class of plants and serves as a practical testbed for comparing classical and modern control strategies. The experimental apparatus and the torsional configurations considered in the modeling are shown in Fig.~\ref{fig:apparatus_intro}.

%==============Guidelines 1-3==================%

\section{System Modeling \& Properties}
\subsection{Guideline 1: System Parameters and Governing Equations}

The experimental plant is a two-degree-of-freedom (2-DOF) torsional system consisting of two rotating disks coupled through a flexible shaft. The physical parameters of the system are provided in the ECP Model~205 laboratory manual \cite{b1} and are summarized in Table~\ref{tab:params}.

\begin{table}[htbp]
    \centering
    \caption{ECP Model 205 Torsional System Parameters}
    \label{tab:params}
    \begin{tabular}{l c c}
        \toprule
        \textbf{Parameter} & \textbf{Symbol} & \textbf{Value} \\
        \midrule
        Disk 1 inertia & $J_1$ & $0.0108~\text{kg}\cdot\text{m}^2$ \\
        Disk 2 inertia & $J_2$ & $0.0103~\text{kg}\cdot\text{m}^2$ \\
        Disk 1 damping coefficient & $c_1$ & $0.007~\text{N}\cdot\text{m}\cdot\text{s/rad}$ \\
        Disk 2 damping coefficient & $c_2$ & $0.001~\text{N}\cdot\text{m}\cdot\text{s/rad}$ \\
        Shaft torsional stiffness & $k_1$ & $1.37~\text{N}\cdot\text{m/rad}$ \\
        Motor voltage–torque gain & $k_{hw}$ & $17.408~\text{N}\cdot\text{m/V}$ \\
        \bottomrule
    \end{tabular}
\end{table}

Applying Newton’s second law for rotational motion to each disk yields the coupled equations of motion
\begin{align}
J_1 \ddot{\theta}_1 + c_1 \dot{\theta}_1 + k_1(\theta_1 - \theta_2) &= T(t),
\label{eq:eom1} \\
J_2 \ddot{\theta}_2 + c_2 \dot{\theta}_2 + k_1(\theta_2 - \theta_1) &= 0,
\label{eq:eom2}
\end{align}
where $\theta_1$ and $\theta_2$ denote the angular displacements of Disk~1 and Disk~2, respectively, and $T(t)=k_{hw}V(t)$ represents the applied motor torque.\\
To derive the state-space representation, define $\mathbf{x} = [\theta_1, \dot{\theta}_1, \theta_2, \dot{\theta}_2]^T$. Solving for accelerations:
\begin{align}
\ddot{\theta}_1 &= -\frac{k_1}{J_1}\theta_1 - \frac{c_1}{J_1}\dot{\theta}_1 + \frac{k_1}{J_1}\theta_2 + \frac{k_{hw}}{J_1}V \\
\ddot{\theta}_2 &= \frac{k_1}{J_2}\theta_1 - \frac{k_1}{J_2}\theta_2 - \frac{c_2}{J_2}\dot{\theta}_2
\end{align}

\textbf{State equations:}
\begin{equation}
\dot{\mathbf{x}} = 
\begin{bmatrix}
0 & 1 & 0 & 0 \\
-\frac{k_1}{J_1} & -\frac{c_1}{J_1} & \frac{k_1}{J_1} & 0 \\
0 & 0 & 0 & 1 \\
\frac{k_1}{J_2} & 0 & -\frac{k_1}{J_2} & -\frac{c_2}{J_2}
\end{bmatrix}
\mathbf{x}
+
\begin{bmatrix}
0 \\ \frac{k_{hw}}{J_1} \\ 0 \\ 0
\end{bmatrix}
T
\label{eq:state_eq}
\end{equation}

\textbf{Output equation:}
\begin{equation}
\mathbf{y} = \begin{bmatrix} \theta_1 \\ \theta_2 \end{bmatrix} = 
\begin{bmatrix}
1 & 0 & 0 & 0 \\
0 & 0 & 1 & 0
\end{bmatrix}
\mathbf{x}
\label{eq:output_eq}
\end{equation}

Substituting numerical values:
\begin{equation}\footnotesize
A = 
\begin{bmatrix}
0 & 1 & 0 & 0 \\
-126.85 & -0.648 & 126.85 & 0 \\
0 & 0 & 0 & 1 \\
133.01 & 0 & -133.01 & -0.097
\end{bmatrix}, \quad
B = 
\begin{bmatrix}
0 \\ 1611.85 \\ 0 \\ 0
\end{bmatrix}
\label{eq:ABCD_numeric}
\end{equation}

The complete numerical state-space model is:
\begin{equation}\footnotesize
\begin{bmatrix}
\dot{x}_1 \\ \dot{x}_2 \\ \dot{x}_3 \\ \dot{x}_4
\end{bmatrix}
=
\begin{bmatrix}
0 & 1 & 0 & 0 \\
-126.85 & -0.648 & 126.85 & 0 \\
0 & 0 & 0 & 1 \\
133.01 & 0 & -133.01 & -0.097
\end{bmatrix}
\begin{bmatrix}
x_1 \\ x_2 \\ x_3 \\ x_4
\end{bmatrix}
+
\begin{bmatrix}
0 \\ 1611.85 \\ 0 \\ 0
\end{bmatrix}
V
\label{eq:state_eq_numerical}
\end{equation}
\subsection{Guideline 2: Derivation of Transfer Functions from State-Space}

The transfer function is derived from the state-space model using:
\begin{equation}
G(s) = C(sI - A)^{-1}B + D
\label{eq:tf_formula}
\end{equation}

Since $D = \mathbf{0}$, we compute $G(s) = C(sI - A)^{-1}B$ directly. Using the matrices from \eqref{eq:ABCD_numeric}:
\begin{equation}
\footnotesize
sI - A = 
\begin{bmatrix}
s & -1 & 0 & 0 \\
126.85 & s+0.648 & -126.85 & 0 \\
0 & 0 & s & -1 \\
-133.01 & 0 & 133.01 & s+0.097
\end{bmatrix}
\end{equation}

The characteristic polynomial (denominator) is found from $\det(sI - A)$:
\begin{equation}
\det(sI - A) = s^4 + 0.745s^3 + 259.9s^2 + 193.3s
\label{eq:char_poly}
\end{equation}

Computing $(sI-A)^{-1}$ requires the adjugate matrix (transpose of cofactors) divided by the determinant. For the transfer functions, we need:
\begin{equation}
G(s) = C \cdot \frac{\text{adj}(sI-A)}{\det(sI-A)} \cdot B
\end{equation}


\begin{equation}
\footnotesize
= \frac{1}{s^4 + 0.745s^3 + 259.9s^2 + 193.3s}
\begin{bmatrix}
1611.85(s^2 + 0.097s) \\
1611.85(133.01s)
\end{bmatrix}
\end{equation}

Expanding the numerators:
With $C = \footnotesize\begin{bmatrix} 1 & 0 & 0 & 0 \\ 0 & 0 & 1 & 0 \end{bmatrix}$ and $B = \footnotesize\begin{bmatrix} 0 \\ 1611.85 \\ 0 \\ 0 \end{bmatrix}$, the multiplication yields:
\begin{align}
G_1(s) &= \frac{\Theta_1(s)}{V(s)} = \frac{1611.85s^2 + 156.5s + 214400}{s^4 + 0.745s^3 + 259.9s^2 + 193.3s}
\label{eq:tf1} \\
G_2(s) &= \frac{\Theta_2(s)}{V(s)} = \frac{214400}{s^4 + 0.745s^3 + 259.9s^2 + 193.3s}
\label{eq:tf2}
\end{align}

Both outputs share the same characteristic polynomial (denominator). The pole at $s=0$ confirms the integrator, while the complex poles at $s \approx -0.37 \pm j16.1$ represent the lightly-damped oscillatory modes.

\subsection{Guideline 3: Controllable, Observable, and Jordan Canonical Forms}

A given state-space realization $(A,B,C,D)$ is not unique---infinitely many equivalent representations exist, related by similarity transformations $\bar{A} = TAT^{-1}$, $\bar{B} = TB$, $\bar{C} = CT^{-1}$. Three important canonical forms are derived below.

\subsubsection{Controllable Canonical Form}

For a SISO system with characteristic polynomial $\det(sI-A) = s^4 + a_3s^3 + a_2s^2 + a_1s + a_0$, the controllable canonical form is:
\begin{equation}
\footnotesize
A_c = 
\begin{bmatrix}
0 & 1 & 0 & 0 \\
0 & 0 & 1 & 0 \\
0 & 0 & 0 & 1 \\
-a_0 & -a_1 & -a_2 & -a_3
\end{bmatrix}, \quad
B_c = 
\begin{bmatrix}
0 \\ 0 \\ 0 \\ 1
\end{bmatrix}
\label{eq:controllable_canonical}
\end{equation}

From the characteristic polynomial \eqref{eq:char_poly} with coefficients $a_3=0.745$, $a_2=259.9$, $a_1=193.3$, $a_0=0$:
\begin{equation}
\footnotesize
A_c = 
\begin{bmatrix}
0 & 1 & 0 & 0 \\
0 & 0 & 1 & 0 \\
0 & 0 & 0 & 1 \\
0 & -193.3 & -259.9 & -0.745
\end{bmatrix}, \quad
B_c = 
\begin{bmatrix}
0 \\ 0 \\ 0 \\ 1
\end{bmatrix}
\end{equation}

The $C_c$ matrix depends on the numerator coefficients of the transfer function (different for $\Theta_1$ and $\Theta_2$). This form is useful for pole-placement controller design.

\subsubsection{Observable Canonical Form}

The observable canonical form is the dual of the controllable form:
\begin{equation}
\footnotesize
A_o = A_c^T = 
\begin{bmatrix}
0 & 0 & 0 & -a_0 \\
1 & 0 & 0 & -a_1 \\
0 & 1 & 0 & -a_2 \\
0 & 0 & 1 & -a_3
\end{bmatrix}, \quad
C_o = B_c^T = \begin{bmatrix}
0 & 0 & 0 & 1
\end{bmatrix}
\label{eq:observable_canonical}
\end{equation}

Substituting our coefficients:
\begin{equation}
\footnotesize
A_o = 
\begin{bmatrix}
0 & 0 & 0 & 0 \\
1 & 0 & 0 & -193.3 \\
0 & 1 & 0 & -259.9 \\
0 & 0 & 1 & -0.745
\end{bmatrix}, \quad
C_o = \begin{bmatrix}
0 & 0 & 0 & 1
\end{bmatrix}
\end{equation}

This form is useful for observer (estimator) design.

\subsubsection{Jordan Canonical Form}

The Jordan form diagonalizes $A$ (or uses Jordan blocks for repeated eigenvalues). The open-loop poles (eigenvalues of $A$) are:
\begin{equation}
\footnotesize
\lambda_1, \lambda_2 = -0.372 \pm j16.113, \quad \lambda_3 = 0, \quad \lambda_4 = -0.001
\end{equation}

Since all eigenvalues are distinct, the Jordan form is diagonal:
\begin{equation}
\footnotesize
A_J = 
\begin{bmatrix}
-0.372+j16.113 & 0 & 0 & 0 \\
0 & -0.372-j16.113 & 0 & 0 \\
0 & 0 & 0 & 0 \\
0 & 0 & 0 & -0.001
\end{bmatrix}
\label{eq:jordan_form}
\end{equation}

The transformation $T$ such that $A_J = TAT^{-1}$ consists of the eigenvectors of $A$ as columns. The matrices $B_J = TB$ and $C_J = CT^{-1}$ are computed accordingly. This form clearly reveals the system's modal structure: two lightly-damped oscillatory modes, one integrator mode (pole at origin), and one fast stable mode.


\subsubsection{Open-Loop Poles and Stability}

The open-loop poles are the roots of the characteristic polynomial in \eqref{eq:char_poly}:
\begin{itemize}
\item $p_{1,2} = -0.183 \pm j16.117$
\item $p_3 = 0$
\item $p_4 = -0.379$
\end{itemize}

The pole at the origin introduces an integrator, rendering the open-loop system marginally stable. The dominant complex-conjugate poles exhibit a natural frequency of approximately $16.1$~rad/s with very low damping ($\zeta \approx 0.011$), resulting in a highly oscillatory response.


% ------------------------------------------------------------------------
% Guideline 4: Open-Loop Time Response
% ------------------------------------------------------------------------
\section{Open-Loop Analysis}

\subsection{Guideline 4: Open-Loop Time Response}

Before designing a controller, we simulated the open-loop system to understand its natural behavior. The system was excited with both impulse and step inputs, under zero and non-zero initial conditions ($\mathbf{x}_0 = [0.1, 0, 0, 0]^T$). The Simulink model used for these simulations is shown in Figure \ref{fig:q4_simulink}.

\begin{figure}[H]
    \centering
    \includegraphics[width=0.95\columnwidth]{Q4-8_Figures/Q4_Simulink_OpenLoop.png}
    \caption{Simulink model for open-loop time response analysis. Four configurations were tested: impulse and step inputs, each with zero and non-zero initial conditions. The state-space block implements $\dot{\mathbf{x}} = A\mathbf{x} + Bu$, $\mathbf{y} = C\mathbf{x}$.}
    \label{fig:q4_simulink}
\end{figure}

\subsubsection{Mathematical Background}
The time-domain response of a linear system is computed by solving the state equation. For the state-space model $\dot{\mathbf{x}} = A\mathbf{x} + Bu$, the general solution is:
\begin{equation}
    \mathbf{x}(t) = e^{At}\mathbf{x}_0 + \int_0^t e^{A(t-\tau)}Bu(\tau)\,d\tau
    \label{eq:state_solution}
\end{equation}

For an \textbf{impulse input} $u(t) = \delta(t)$, the Laplace transform approach gives:
\begin{equation}
    Y(s) = G(s) \cdot 1 = G(s) \quad \Rightarrow \quad y(t) = \mathcal{L}^{-1}\{G(s)\}
\end{equation}
In MATLAB/Simulink, this is computed using the \texttt{impulse()} function or by applying a narrow pulse approximation.

For a \textbf{step input} $u(t) = u_s(t)$, where $U(s) = 1/s$:
\begin{equation}
    Y(s) = G(s) \cdot \frac{1}{s} \quad \Rightarrow \quad y(t) = \mathcal{L}^{-1}\left\{\frac{G(s)}{s}\right\}
\end{equation}

Since our system has a pole at the origin ($s = 0$), the step response will contain a ramp component; this is why the output grows unboundedly.

\subsubsection{Impulse Response}
The impulse response reveals how the system reacts to an instantaneous input. As shown in Figures \ref{fig:q4_impulse_zero} and \ref{fig:q4_impulse_nz}, the output grows unboundedly over time. This is a direct consequence of the integrator pole at the origin ($p_4 = 0$), which causes the system to accumulate the effect of the impulse indefinitely.

\begin{figure}[H]
    \centering
    \includegraphics[width=0.85\columnwidth]{Q4-8_Figures/Q4_1_Impulse_Zero_IC.png}
    \caption{Open-loop impulse response with zero initial conditions. The output grows unboundedly due to the integrator pole. The inset shows the initial transient where the complex poles ($p_{1,2} \approx -0.18 \pm 16.1j$) cause high-frequency oscillations.}
    \label{fig:q4_impulse_zero}
\end{figure}

\begin{figure}[H]
    \centering
    \includegraphics[width=0.85\columnwidth]{Q4-8_Figures/Q4_2_Impulse_NonZero_IC.png}
    \caption{Open-loop impulse response with non-zero initial conditions ($\theta_1(0) = 0.1$ rad). The zoomed inset shows small oscillations before $t=1$s caused by the initial displacement of Disk 1. The overall unbounded behavior is identical to the zero-IC case.}
    \label{fig:q4_impulse_nz}
\end{figure}

The key observation here is that the initial conditions only affect the early transient---the small oscillations visible in the zoomed inset of Figure \ref{fig:q4_impulse_nz}. Once the impulse is applied, the integrator dominates and the response diverges regardless of the starting point.

\subsubsection{Step Response}
The step response is even more dramatic. As shown in Figures \ref{fig:q4_step_zero} and \ref{fig:q4_step_nz}, when a unit step is applied at $t=1$s, the output ramps up to over 350 rad within just one second. This behavior is characteristic of an integrator: a constant input produces a ramp output.

\begin{figure}[H]
    \centering
    \includegraphics[width=0.85\columnwidth]{Q4-8_Figures/Q4_3_Step_Zero_IC.png}
    \caption{Open-loop step response with zero initial conditions. The unit step at $t=1$s causes the output to ramp unboundedly, reaching approximately 350 rad by $t=2$s. This confirms the presence of an integrator in the plant.}
    \label{fig:q4_step_zero}
\end{figure}

\begin{figure}[H]
    \centering
    \includegraphics[width=0.85\columnwidth]{Q4-8_Figures/Q4_4_Step_NonZero_IC.png}
    \caption{Open-loop step response with non-zero initial conditions. The zoomed inset shows oscillations from the initial displacement before the step is applied. The ramp behavior after $t=1$s is identical to the zero-IC case.}
    \label{fig:q4_step_nz}
\end{figure}

These results clearly demonstrate why feedback control is essential for this system. Without it, any sustained input will cause the disk positions to grow without bound---the system is marginally stable at best.

% ------------------------------------------------------------------------
% Guideline 5: Bode Plot and Root Locus
% ------------------------------------------------------------------------
\subsection{Guideline 5: Frequency Response and Root Locus}

To better understand the system's dynamics and prepare for controller design, we analyzed the frequency response (Bode plot) and root locus of the uncompensated plant $G_2(s) = \Theta_2(s)/U(s)$.

\subsubsection{Bode Plot Theory}
The Bode plot displays $|G(j\omega)|$ (magnitude in dB) and $\angle G(j\omega)$ (phase in degrees) versus frequency $\omega$. For stability analysis, two key metrics are computed:
\begin{itemize}
    \item \textbf{Gain Margin (GM)}: The factor by which loop gain can increase before instability. Computed at the \emph{phase crossover frequency} $\omega_{pc}$ where $\angle G(j\omega_{pc}) = -180°$:
    \begin{equation}
        GM = \frac{1}{|G(j\omega_{pc})|}
    \end{equation}
    \item \textbf{Phase Margin (PM)}: The additional phase lag the system can tolerate. Computed at the \emph{gain crossover frequency} $\omega_{gc}$ where $|G(j\omega_{gc})| = 1$:
    \begin{equation}
        PM = 180° + \angle G(j\omega_{gc})
    \end{equation}
\end{itemize}

\subsubsection{Root Locus Theory}
The root locus shows how closed-loop poles move as the loop gain $K$ varies from 0 to $\infty$. For a system with open-loop transfer function $KG(s)$, the closed-loop poles satisfy:
\begin{equation}
    1 + KG(s) = 0 \quad \Rightarrow \quad G(s) = -\frac{1}{K}
\end{equation}
This means the closed-loop poles lie on paths where $|G(s)| = 1/K$ and $\angle G(s) = \pm 180°$ (the \emph{angle condition}). MATLAB's \texttt{rlocus()} function numerically traces these paths.

\subsubsection{Bode Plot Analysis}
Figure \ref{fig:q5_bode} shows the Bode diagram of the open-loop plant. Several key features stand out:

\begin{figure}[H]
    \centering
    \includegraphics[width=0.9\columnwidth]{Q4-8_Figures/Q5_1_Bode_Uncompensated.png}
    \caption{Bode plot of the uncompensated system $G_2(s)$. The resonant peak near 16 rad/s corresponds to the lightly-damped complex poles. The phase starts at $-90°$ due to the integrator pole at the origin. The gain margin is \textbf{negative} ($-22.1$ dB), indicating the closed-loop system would be unstable with unity feedback.}
    \label{fig:q5_bode}
\end{figure}

\begin{itemize}
    \item \textbf{Low-frequency behavior}: The magnitude starts high ($\sim$100 dB) and rolls off at $-20$ dB/decade, while the phase begins at $-90°$. This is the signature of an integrator.
    \item \textbf{Resonant peak}: Around 16 rad/s, there is a sharp resonant peak caused by the lightly-damped complex poles at $p_{1,2} = -0.18 \pm 16.1j$. The very low damping ratio ($\zeta \approx 0.01$) makes this peak particularly pronounced.
    \item \textbf{Gain margin}: The gain margin is $-22.1$ dB at $\omega_{pc} = 11.5$ rad/s. A \emph{negative} gain margin means the system is already unstable with unity feedback---we would need to \emph{reduce} the loop gain significantly just to achieve marginal stability.
\end{itemize}

\subsubsection{Root Locus Analysis}
The root locus in Figure \ref{fig:q5_rlocus} shows how the closed-loop poles move as the loop gain increases. The open-loop poles are:
\begin{itemize}
    \item $p_{1,2} = -0.18 \pm 16.12j$ --- complex conjugate pair (oscillatory modes)
    \item $p_3 = -0.38$ --- stable real pole
    \item $p_4 = 0$ --- integrator at the origin
\end{itemize}

\begin{figure}[H]
    \centering
    \includegraphics[width=0.9\columnwidth]{Q4-8_Figures/Q5_2_RootLocus_Uncompensated.png}
    \caption{Root locus of the open-loop system. The four branches originate from the open-loop poles (red X markers). The grid lines show the design targets: $\zeta = 0.69$ and $\omega_n = 3.33$ rad/s. Notice that the branches from the complex poles head toward the right-half plane, confirming the instability observed in the Bode analysis.}
    \label{fig:q5_rlocus}
\end{figure}

The root locus reveals that simply adding proportional gain will not stabilize this system effectively. The branches from the complex poles move toward the right-half plane, which explains the negative gain margin. A more sophisticated controller (PID) is needed to reshape the root locus and achieve the desired closed-loop pole locations.

\begin{figure}[H]
    \centering
    \includegraphics[width=0.85\columnwidth]{Q4-8_Figures/Q5_3_PoleZero_Map.png}
    \caption{Pole-zero map of the open-loop system showing all four poles with their exact locations. The pole at the origin ($p_4 = 0$) is responsible for the integrator behavior observed in the time-domain responses.}
    \label{fig:q5_pzmap}
\end{figure}

% ------------------------------------------------------------------------
% Guideline 6: PID Controller Design
% ------------------------------------------------------------------------
\section{Classical Controller Design (PID)}
\subsection{Guideline 6: PID Controller Design}

Based on the analysis in the previous sections, we designed a PID controller to stabilize the system and achieve reasonable tracking performance. 
Using the standard second-order approximation, we get:
\begin{equation}
    \zeta = \frac{-\ln(0.05)}{\sqrt{\pi^2 + \ln^2(0.05)}} \approx 0.69
\end{equation}
\begin{equation}
    \omega_n = \frac{4.6}{\zeta T_s} = \frac{4.6}{0.69 \times 2.0} \approx 3.33 \text{ rad/s}
\end{equation}

The design targets were:
\begin{itemize}
    \item Percent Overshoot: $PO \leq 5\%$ (corresponding to $\zeta \geq 0.69$)
    \item Settling Time: $T_s \approx 2$ seconds (corresponding to $\omega_n \approx 3.33$ rad/s)
\end{itemize}

\subsubsection{Design Approach}
The PID controller has the transfer function:
\begin{equation}
    C(s) = K_p + \frac{K_i}{s} + K_d s = \frac{K_d s^2 + K_p s + K_i}{s}
\end{equation}

As we know, each term serves a specific purpose in shaping the closed-loop response:
\begin{itemize}
    \item \textbf{Proportional ($K_p$)}: Provides instantaneous correction proportional to error. Higher $K_p$ reduces steady-state error but can increase overshoot.
    \item \textbf{Integral ($K_i/s$)}: Accumulates past errors, ensuring zero steady-state error for step inputs. Adds a pole at $s = 0$.
    \item \textbf{Derivative ($K_d s$)}: Anticipates future error by acting on the rate of change. Adds damping but amplifies high-frequency noise.
\end{itemize}

\subsubsection{Root Locus Design Procedure}
We used the root locus technique to tune the gains. The PID controller can be rewritten to expose its zeros:
\begin{equation}\footnotesize
    C(s) = K_d \cdot \frac{s^2 + (K_p/K_d)s + (K_i/K_d)}{s} = K_d \cdot \frac{(s - z_1)(s - z_2)}{s}
\end{equation}
where $z_1$ and $z_2$ are the controller zeros. The idea was to place these zeros to ``pull'' the root locus branches into the left-half plane.

Starting with a PD compensator of the form $(s/3 + 1)$, we used MATLAB's \texttt{evalfr()} function to compute the required gain at a desired pole location $s^* = -\sigma + j\omega_d$:
\begin{equation}
    K = \frac{1}{|G_{compensated}(s^*)|}
\end{equation}
This satisfies the magnitude condition of the root locus at $s^*$.

However, we quickly discovered that achieving the aggressive design specs ($\zeta = 0.69$, $\omega_n = 3.33$) with a PID controller on this plant was extremely difficult. The lightly-damped resonant modes at $\pm 16.1j$ made the system very sensitive to gain changes. Higher gains caused the closed-loop poles to move toward instability.

We also attempted to use MATLAB's \texttt{pidTuner} auto-tuning tool, but even the automated optimization could not find gains that met our specifications while maintaining stability. This limitation is fundamental to the PID structure---it simply does not have enough degrees of freedom to independently control all four closed-loop poles.

\subsubsection{Final PID Gains}
After extensive manual tuning, we settled on the following conservative gains:
\begin{equation}\small
    K_p = 2.8 \times 10^{-4}, \quad K_i = 3.0 \times 10^{-6}, \quad K_d = 1.0 \times 10^{-4}
\end{equation}

These gains are deliberately small to avoid exciting the resonant modes. The trade-off is slower response and larger overshoot than originally targeted.

% ------------------------------------------------------------------------
% Guideline 7: PID Controller Responses
% ------------------------------------------------------------------------
\subsection{Guideline 7: PID Controller Time-Domain Responses}

The designed PID controller was tested with step, square wave, and sinusoidal inputs. The closed-loop Simulink model is shown in Figure \ref{fig:q7_simulink}.

\begin{figure}[H]
    \centering
    \includegraphics[width=0.95\columnwidth]{Q4-8_Figures/Q7_Simulink_PID_Controller.png}
    \caption{Simulink model for PID closed-loop control. The three parallel branches implement proportional ($K_p$), integral ($K_i/s$), and derivative ($K_d s$) actions. The feedback loop closes around the load position $\theta_2$.}
    \label{fig:q7_simulink}
\end{figure}

\subsubsection{Step Response}
Figure \ref{fig:q7_step} shows the closed-loop step response. The system successfully tracks the step input, but the performance differs from our design targets:
\begin{itemize}
    \item \textbf{Overshoot}: Approximately 20\% (target was 5\%)
    \item \textbf{Settling time}: Approximately 15--20 seconds (target was 2 seconds)
    \item \textbf{Steady-state error}: Zero (the integral action eliminates the error)
\end{itemize}

\begin{figure}[H]
    \centering
    \includegraphics[width=0.9\columnwidth]{Q4-8_Figures/Q7_1_PID_Step_Response.png}
    \caption{PID closed-loop step response. The controller stabilizes the system and achieves zero steady-state error, but with higher overshoot ($\sim$20\%) and longer settling time ($\sim$15s) than the design targets. The oscillations during the transient are caused by the lightly-damped resonant modes that the PID cannot fully suppress.}
    \label{fig:q7_step}
\end{figure}

The slower-than-expected response is a direct consequence of our conservative gain selection. Higher gains caused instability during testing, so we accepted slower performance in exchange for robustness.

\subsubsection{Square Wave Response}
The square wave input (Figure \ref{fig:q7_square}) reveals the bandwidth limitations of our controller. The system cannot track the sharp transitions of the square wave---instead, it produces a smoothed, almost sinusoidal response.

\begin{figure}[H]
    \centering
    \includegraphics[width=0.9\columnwidth]{Q4-8_Figures/Q7_2_PID_Square_Response.png}
    \caption{PID response to a 0.1 Hz square wave input. The controller acts as a low-pass filter, smoothing out the sharp transitions. The output never reaches the commanded 1 rad or 0 rad levels within each half-period, indicating limited bandwidth.}
    \label{fig:q7_square}
\end{figure}

This behavior makes sense: our low PID gains result in a closed-loop bandwidth that is too narrow to track the high-frequency content of a square wave.

\subsubsection{Sinusoidal Response}
The sinusoidal tracking test (Figure \ref{fig:q7_sine}) at 1 rad/s further confirms the bandwidth limitation:
\begin{itemize}
    \item \textbf{Amplitude attenuation}: The output amplitude is only about 0.3 rad compared to the 1 rad input---a 70\% reduction.
    \item \textbf{Phase lag}: There is visible phase delay between the input and output.
\end{itemize}

\begin{figure}[H]
    \centering
    \includegraphics[width=0.9\columnwidth]{Q4-8_Figures/Q7_3_PID_Sine_Response.png}
    \caption{PID response to a 1 rad/s sinusoidal input. The output is significantly attenuated (amplitude $\sim$0.3 rad vs. 1 rad input) and phase-lagged. This confirms that the closed-loop bandwidth is below 1 rad/s---a consequence of the conservative gain selection.}
    \label{fig:q7_sine}
\end{figure}

% ------------------------------------------------------------------------
% Guideline 8: Robustness Analysis
% ------------------------------------------------------------------------
\subsection{Guideline 8: Robustness to Measurement Noise}

A practical concern with PID controllers is their sensitivity to measurement noise, particularly the derivative term which amplifies high-frequency noise. We tested the controller's robustness by adding white noise to the feedback signal. The Simulink model for noise injection is shown in Figure \ref{fig:q8_simulink}.

\begin{figure}[H]
    \centering
    \includegraphics[width=0.95\columnwidth]{Q4-8_Figures/Q8_Simulink_Noise_Test.png}
    \caption{Simulink model for noise robustness testing. Band-limited white noise is added to the feedback signal before entering the PID controller. The ``pid\_small\_noise\_feedback'' and ``pid\_small\_noise\_response'' blocks log the corrupted measurement and actual plant output, respectively.}
    \label{fig:q8_simulink}
\end{figure}

\subsubsection{Small Noise Test}
With a noise power of $5 \times 10^{-7}$, the controller maintains acceptable performance (Figure \ref{fig:q8_small}). The measured feedback signal shows visible noise, but the actual plant output remains relatively clean. The integral action helps filter the noise over time.

\begin{figure}[H]
    \centering
    \includegraphics[width=0.85\columnwidth]{Q4-8_Figures/Q8_1_PID_SmallNoise_Combined.png}
    \caption{PID controller response with small measurement noise (power = $5 \times 10^{-7}$). (A) The noisy feedback signal fed to the controller. (B) The actual plant output, which remains relatively clean despite the noisy measurement. The controller successfully rejects this level of noise.}
    \label{fig:q8_small}
\end{figure}

\subsubsection{Large Noise Test}
With a noise power of 0.001, the controller fails dramatically (Figure \ref{fig:q8_large}). The noisy feedback causes the controller to generate erratic control signals, which in turn cause the actual plant output to deviate significantly from the reference.

\begin{figure}[H]
    \centering
    \includegraphics[width=0.85\columnwidth]{Q4-8_Figures/Q8_2_PID_LargeNoise_Combined.png}
    \caption{PID controller response with large measurement noise (power = 0.001). (A) The severely corrupted feedback signal. (B) The actual plant output, which fails to track the reference and shows large oscillations. This demonstrates a fundamental limitation of PID control---it cannot distinguish between actual system dynamics and measurement noise.}
    \label{fig:q8_large}
\end{figure}

\subsubsection{Discussion}
The robustness tests reveal an important lesson: \textbf{the derivative term in a PID controller is a double-edged sword}. While it improves transient response by anticipating future errors, it also amplifies high-frequency noise. In a real implementation, a derivative filter (e.g., $K_d s / (\tau s + 1)$) would be essential to limit the high-frequency gain.

The large noise test also highlights why modern control techniques like state-feedback with observers (covered in the next section) can be advantageous---they provide a systematic way to filter noise while maintaining good tracking performance.

\subsubsection{Summary of Classical Control Design}
The PID controller successfully stabilized the marginally-stable open-loop plant and achieved zero steady-state error for step inputs. However, several limitations were observed:
\begin{enumerate}
    \item The transient performance (overshoot, settling time) did not meet the original design targets.
    \item The closed-loop bandwidth was limited, resulting in poor tracking of fast-varying signals.
    \item The controller was sensitive to measurement noise, particularly at higher noise levels.
\end{enumerate}

These limitations motivate the use of modern state-space control techniques, which offer more degrees of freedom for shaping the closed-loop dynamics. 

\section{Modern Controller Design (State-Feedback)}
This section presents the design, simulation, and validation of a full-state feedback controller and a Luenberger observer. The design objective is to arbitrarily place the closed-loop poles to achieve specific time-domain performance metrics, specifically dealing with the system's resonant modes more effectively than the classical approach.

% ------------------------------------------------------------------------
% Guideline 9: State Feedback Design
% ------------------------------------------------------------------------
\subsection{Guideline 9: State-Feedback Control Law}
\subsubsection{Controllability Analysis}

Before designing a state-feedback controller, we must verify that the system is \textit{controllable}, that arbitrary pole placement is possible. A system $(A,B)$ is controllable if and only if the controllability matrix has full rank.

The controllability matrix is defined as:
\begin{equation}
\mathcal{C} = \begin{bmatrix} B & AB & A^2B & A^3B \end{bmatrix}
\label{eq:controllability_matrix}
\end{equation}

For our fourth-order system, $\mathcal{C}$ must be $4 \times 4$ with rank 4. Computing each term:
\begin{equation}\footnotesize
B = \begin{bmatrix} 0 \\ 1611.85 \\ 0 \\ 0 \end{bmatrix}, \quad
AB = \begin{bmatrix} 1611.85 \\ -1044.25 \\ 0 \\ 214324.4 \end{bmatrix}
\end{equation}
\begin{equation}\footnotesize
A^2B = \begin{bmatrix} -1044.25 \\ 27254857 \\ 214324.4 \\ -28502916 \end{bmatrix}, \quad
A^3B = \begin{bmatrix} 27254857 \\ -3483299200 \\ -28502916 \\ 3792732600 \end{bmatrix}
\end{equation}

The full controllability matrix is:
\begin{equation}\footnotesize
\mathcal{C} = 
\begin{bmatrix}
0 & 1611.85 & -1044.25 & 27254857 \\
1611.85 & -1044.25 & 27254857 & -3483299200 \\
0 & 0 & 214324.4 & -28502916 \\
0 & 214324.4 & -28502916 & 3792732600
\end{bmatrix}
\end{equation}

Using MATLAB's \texttt{rank()} function:
\begin{verbatim}
rank(C) = 4
\end{verbatim}

Since $\text{rank}(\mathcal{C}) = 4 = n$ (system order), the system is \textbf{fully controllable}. This confirms that we can arbitrarily place all four closed-loop poles using state-feedback control $u = -K\mathbf{x}$. \\

\subsubsection{Controller Design}
The control law is defined as $u = -K\mathbf{x} + \bar{N}r$. The gain matrix $K$ is designed to meet the following specifications:
\begin{itemize}
    \item Percent Overshoot ($PO$) $\approx 5\%$
    \item Settling Time ($T_s$) $\approx 2.0$ seconds
\end{itemize}

from Equation (\ref{eq:U_theta2})the required damping ratio ($\zeta$) and natural frequency ($\omega_n$) are 

\begin{equation}
    \zeta \geq 0.69\space\space\space\space \& \space\space\space\space
    \omega_n  \approx 3.33 \text{ rad/s}
\end{equation}

To implement this, the dominant closed-loop poles were placed at:
\begin{equation}
    p_{1,2} = -\zeta\omega_n \pm j\omega_n\sqrt{1-\zeta^2} \approx -2.31 \pm j2.39
\end{equation}
To minimize their impact on the transient response, the remaining two non-dominant poles were placed $10\times$ further to the left in the complex plane at $p_{3,4} \approx -23.1$.

\subsubsection{MATLAB Implementation}
The gain matrix $K$ was computed using the MATLAB \texttt{place(A,B,p)} function, which numerically solves the pole placement problem (Ackermann's formula). The resulting state-feedback gain matrix is:
\begin{equation}
    K = 
    \begin{bmatrix} 
    5.5561 & 0.5492 & -5.0580 & -0.3127 
    \end{bmatrix}
\end{equation}

To ensure unity steady-state gain for a step input ($y_{ss} = r$), the feedforward gain $\bar{N}$ was calculated as:
\begin{equation}
    \bar{N} = \frac{1}{G_{closed-loop}(0)} = 0.4980
\end{equation}

% ------------------------------------------------------------------------
% Guideline 10: State Feedback Response
% ------------------------------------------------------------------------
\subsection{Guideline 10: State-Feedback Simulation Results}

The state-feedback controller was simulated using the Simulink model shown in Figure \ref{fig:simulink_sf}. The system was tested against Step, Square, and Sinusoidal inputs.

\begin{figure}[htbp]
    \centering
    \includegraphics[width=0.9\columnwidth]{Figure Part 9-11/Simulink 9-12/part10_simulink_diagrahm.png}
    \caption{Simulink model used for state-feedback simulation.}
    \label{fig:simulink_sf}
\end{figure}

\subsubsection{Response to Initial Conditions}
To verify the regulator properties, the system was simulated with an arbitrary initial displacement $\mathbf{x}_0 = [0.1, 0, -0.1, 0]^T$ (twisting the disks in opposite directions) with zero reference input.

\begin{figure}[H]
    \centering
    \includegraphics[width=0.9\columnwidth]{Figure Part 9-11/Figures/Part9_step_input.png}
    \caption{Zero-input response to initial conditions $\mathbf{x}_0 = [0.1, 0, -0.1, 0]^T$.}
    \label{fig:sf_initial}
\end{figure}

As shown in Figure \ref{fig:sf_initial}, the controller successfully drives the initial error in $\theta_1$ (0.1 rad) to zero within approximately 1.9 seconds. This effectively meets the 2.0-second settling time requirement and confirms the system's stability.

\subsubsection{Tracking Performance}
The tracking performance for various reference inputs is presented below.

\begin{figure}[H]
    \centering
    \includegraphics[width=0.9\columnwidth]{Figure Part 9-11/Figures/Part9_step_big_init.png}
    \caption{Closed-loop Step Response (State-Feedback).}
    \label{fig:sf_step}
\end{figure}

\textbf{Analysis:} The step response (Fig. \ref{fig:sf_step}) shows a rise time consistent with the design frequency $\omega_n = 3.33$ rad/s. The overshoot is slightly higher than the expected 5\%, which is attributed to the influence of the plant zeros that were not accounted for in the all-pole approximation used for the design.

\begin{figure}[H]
    \centering
    \includegraphics[width=0.9\columnwidth]{Figure Part 9-11/Figures/Part10_squareWave_input.png}
    \caption{Closed-loop Square Wave Response.}
    \label{fig:sf_square}
\end{figure}

\begin{figure}[H]
    \centering
    \includegraphics[width=0.9\columnwidth]{Figure Part 9-11/Figures/Part10_SineWave_input_smaller_init_cond.png}
    \caption{Closed-loop Sinusoidal Response (4 rad/s).}
    \label{fig:sf_sine}
\end{figure}

\textbf{Analysis:} In the sinusoidal response (Fig. \ref{fig:sf_sine}), the system tracks the 1 rad/s input with unity magnitude ratio. A small phase lag is observed, which is expected for a Type 0 system tracking a time-varying signal; however, the output remains stable and smooth.

% ------------------------------------------------------------------------
% Guideline 11: Observer Design
% ------------------------------------------------------------------------
\subsection{Guideline 11: Observer Design \& Simulation}

In practical applications, measuring all state variables (specifically angular velocities) is often noisy or expensive. Therefore, an observer is designed to estimate the state vector $\hat{\mathbf{x}}$ from the available position measurements $\mathbf{y} = [\theta_1, \theta_2]^T$.
\subsubsection{Observability Analysis}

Before designing an observer, we must verify that the system is \textit{observable}, that the internal states can be reconstructed from the output measurements. A system $(A,C)$ is observable if and only if the observability matrix has full rank.

The observability matrix is defined as:
\begin{equation}\footnotesize
\mathcal{O} = 
\begin{bmatrix}
C \\ CA \\ CA^2 \\ CA^3
\end{bmatrix}
\label{eq:observability_matrix}
\end{equation}

Computing each term:
\begin{equation}\footnotesize
C = \begin{bmatrix} 1 & 0 & 0 & 0 \\ 0 & 0 & 1 & 0 \end{bmatrix}
\end{equation}
\begin{equation}\footnotesize
CA = \begin{bmatrix} 0 & 1 & 0 & 0 \\ 0 & 0 & 0 & 1 \end{bmatrix}
\end{equation}
\begin{equation}\footnotesize
CA^2 = \begin{bmatrix} -126.85 & -0.648 & 126.85 & 0 \\ 133.01 & 0 & -133.01 & -0.097 \end{bmatrix}
\end{equation}
\begin{equation}\footnotesize
CA^3 = \begin{bmatrix} 82.22 & 164.61 & -82.22 & -0.648 \\ 0 & -175.52 & 0 & 12.90 \end{bmatrix}
\end{equation}

The full observability matrix is:
\begin{equation}\footnotesize
\mathcal{O} = 
\begin{bmatrix}
1 & 0 & 0 & 0 \\
0 & 0 & 1 & 0 \\
0 & 1 & 0 & 0 \\
0 & 0 & 0 & 1 \\
-126.85 & -0.648 & 126.85 & 0 \\
133.01 & 0 & -133.01 & -0.097 \\
82.22 & 164.61 & -82.22 & -0.648 \\
0 & -175.52 & 0 & 12.90
\end{bmatrix}
\end{equation}

Using MATLAB's \texttt{rank()} function:
\begin{verbatim}
rank(O) = 4
\end{verbatim}

Since $\text{rank}(\mathcal{O}) = 4 = n$ (system order), the system is \textbf{fully observable}. This confirms that we can estimate all four internal states from the two position measurements $[\theta_1, \theta_2]^T$ using an observer.
\subsubsection{Full-Order Observer Implementation}
A **Full-Order Luenberger Observer** was first designed to estimate the complete state vector. The observer dynamics are governed by:
\begin{equation}
    \dot{\hat{\mathbf{x}}} = (A - LC)\hat{\mathbf{x}} + Bu + L\mathbf{y}
\end{equation}
To ensure the estimation error $e(t) = x(t) - \hat{x}(t)$ converges to zero significantly faster than the controller dynamics, the observer poles were placed at locations $4\times$ faster (further left in the complex plane) than the controller poles.

The calculated observer gain matrix $L = G$ is:
\begin{equation}\footnotesize
    L = 
    \begin{bmatrix} 
    103.2 & 4.7 \\
    743.7 & 415.5 \\
    -30.0 & 102.5 \\
    -2870.9 & 596.3 
    \end{bmatrix}
\end{equation}

The combined Controller-Observer system was simulated using the model in Figure \ref{fig:simulink_obs}. \\

Also Please Note that in the Following Simulink the C and D matric of the Plant were set to 0.

\begin{figure}[htbp]
    \centering
    \includegraphics[width=0.9\columnwidth]{Figure Part 9-11/Simulink 9-12/Part_11_simulink_diagrahm.png}
    \caption{Simulink model including the Plant and the Full-Order Observer.}
    \label{fig:simulink_obs}
\end{figure}

The tracking performance using the estimated states $\hat{\mathbf{x}}$ is shown below.

\begin{figure}[H]
    \centering
    \includegraphics[width=0.9\columnwidth]{Figure Part 9-11/Figures/part_11_step_input.png}
    \caption{Step Response using Full-Order Observer States.}
    \label{fig:obs_step}
\end{figure}

\textbf{Analysis:} In Figure \ref{fig:obs_step}, a transient "spike" is visible at $t=0$. This occurs because the observer is initialized at zero ($\hat{x}_0 = 0$) while the plant has non-zero initial conditions. The error dynamics converge rapidly (within approx. 0.5s), after which the system tracks the reference perfectly.

\begin{figure}[H]
    \centering
    \includegraphics[width=0.9\columnwidth]{Figure Part 9-11/Figures/part_11_square_input.png}
    \caption{Square Wave Response using Full-Order Observer.}
    \label{fig:obs_square}
\end{figure}

\begin{figure}[H]
    \centering
    \includegraphics[width=0.9\columnwidth]{Figure Part 9-11/Figures/part_11_sine_input.png}
    \caption{Sinusoidal Response using Full-Order Observer.}
    \label{fig:obs_sine}
\end{figure}

% ------------------------------------------------------------------------
% Reduced Order Section
% ------------------------------------------------------------------------
\subsubsection{Reduced-Order Observer Implementation}
As per Guideline 11, a Reduced-Order Observer was also designed. Since the positions $\mathbf{x}_a = [\theta_1, \theta_2]^T$ are directly measured, it is theoretically sufficient to estimate only the unmeasured velocities $\mathbf{x}_b = [\dot{\theta}_1, \dot{\theta}_2]^T$.

To design this, the state vector is partitioned into measured ($a$) and unmeasured ($b$) components:
\begin{equation}
    \dot{\mathbf{x}} = 
    \begin{bmatrix} A_{aa} & A_{ab} \\ A_{ba} & A_{bb} \end{bmatrix} 
    \begin{bmatrix} \mathbf{x}_a \\ \mathbf{x}_b \end{bmatrix} + 
    \begin{bmatrix} B_a \\ B_b \end{bmatrix} u
\end{equation}

The dynamics of the reduced-order estimate $\hat{\mathbf{x}}_b$ are derived to be:
\begin{equation}
    \dot{\hat{\mathbf{x}}}_b = (A_{bb} - L_{red}A_{ab})\hat{\mathbf{x}}_b + A_{ba}\mathbf{x}_a + B_b u + L_{red}(\dot{\mathbf{x}}_a - A_{aa}\mathbf{x}_a)
\end{equation}

In implementation, a change of variables is often used to avoid differentiating the noisy measurement $\mathbf{x}_a$. The resulting step response for the reduced-order system is shown below.

\begin{figure}[H]
    \centering
    \includegraphics[width=0.9\columnwidth]{Figure Part 9-11/reduced_order step response.png}
    \caption{Step Response using Reduced-Order Observer States.}
    \label{fig:red_obs_step}
\end{figure}

As shown in Figure \ref{fig:red_obs_step}, the reduced-order observer achieves similar tracking performance to the full-order design under ideal conditions, successfully stabilizing the system.

% ------------------------------------------------------------------------
% Robustness Section
% ------------------------------------------------------------------------
\subsubsection{Robustness Testing \& Comparison}
To evaluate the robustness of both designs, a noise injection test was performed. Band-limited white noise (Power $= 5 \times 10^{-7}$) was added to the plant output measurements $y$ to simulate sensor quantization and electromagnetic interference. We also added the same noise but on the order of 0.001 so see more dramatic effect on both full order and reduced order.

\begin{figure}[htbp]
    \centering
    \includegraphics[width=0.9\columnwidth]{Figure Part 9-11/Simulink_noise_diagram.png}
    \caption{Simulink setup for noise injection testing.}
    \label{fig:noise_diagram}
\end{figure}

The system response under noise for both observer types is presented below.

\begin{figure}[H]
    \centering
    \includegraphics[width=0.9\columnwidth]{Figure Part 9-11/n_full_order_step_response_with_noise.png}
    \caption{Full-Order Observer Response with Measurement Noise.}
    \label{fig:full_noise}
\end{figure}

\begin{figure}[H]
    \centering
    \includegraphics[width=0.9\columnwidth]{Figure Part 9-11/n_reduced_order_step_response_with_noise.png}
    \caption{Reduced-Order Observer Response with Measurement Noise.}
    \label{fig:red_noise}
\end{figure}


Now testing with bigger noise 0.001 to clearly see the impact and stability

\begin{figure}[H]
    \centering
    \includegraphics[width=0.9\columnwidth]{Figure Part 9-11/np_full_order_step_response_with_noise.png}
    \caption{Full-Order Observer Response with Measurement Noise.}
    \label{fig:full_noise}
\end{figure}

\begin{figure}[H]
    \centering
    \includegraphics[width=0.9\columnwidth]{Figure Part 9-11/np_reduced_order_step_response_with_noise.png}
    \caption{Reduced-Order Observer Response with Measurement Noise.}
    \label{fig:red_noise}
\end{figure}



\textbf{Robustness Comparative Analysis FO vs RO:} 
Both observers demonstrated robust performance, maintaining stability and tracking the reference step despite the noisy measurements. As observed in Figures \ref{fig:full_noise} and \ref{fig:red_noise}, the impact of the noise on the final load position ($\theta_2$) was negligible in both casesfor the normal noise and even with the exaggerated noise, the system remained stable.

However, a theoretical distinction exists between the two:
\begin{itemize}
    \item \textbf{Full-Order Observer:} Utilizes the system model to estimate \textit{all} states. It effectively acts as a low-pass filter, smoothing out the high-frequency measurement noise before it enters the control loop.
    \item \textbf{Reduced-Order Observer:} Assumes the measured states ($\theta_1, \theta_2$) are perfect and feeds them directly into the state vector. While computationally more efficient (Order 2 vs. Order 4), it lacks the inherent filtering property for the position states.
\end{itemize}

For this application, the Full-Order observer is preferred as it provides a cleaner internal state estimate without relying on the assumption of noise-free sensors. But they are practically the same, so both could be used.






% ------------------------------------------------------------------------
% Guideline 12: Compensator Analysis
% ------------------------------------------------------------------------
\subsection{Guideline 12: Compensator Transfer Function}

The combination of the state-feedback law $u = -K\hat{\mathbf{x}}$ and the observer constitutes a single dynamic compensator $C(s)$. This compensator processes the measurements $\mathbf{y} = [\theta_1, \theta_2]^T$ to produce the control signal $u$ (Voltage).

\subsubsection{Mathematical Derivation}
Defining the compensator state matrix $A_{comp} = A - BK - LC$, input matrix $B_{comp} = L$, and output matrix $C_{comp} = -K$, the transfer function matrix is:
\begin{equation}
    C(s) = \frac{U(s)}{\mathbf{Y}(s)} = C_{comp}(sI - A_{comp})^{-1}B_{comp}
\end{equation}

Since the system has two measured outputs, the total control law is:
\begin{equation}
    U(s) = C_1(s)\Theta_1(s) + C_2(s)\Theta_2(s)
\end{equation}

Based on the design parameters, the resulting transfer functions from the Motor Position ($\theta_1$) and Load Position ($\theta_2$) to the Control Voltage ($u$) are:

\begin{align}
    \frac{U(s)}{\Theta_1(s)} &= 
    \frac{-2031.4 \left(s + 95.3\right)\left(s^2 + 3.02s + 41.8\right)}
         {\left(s - 3.8\right)\left(s + 95.3\right)\left(s^2 + 165.8s + 8643\right)}
    \label{eq:U_theta1}
    \\[6pt]
    \frac{U(s)}{\Theta_2(s)} &= 
    \frac{450.3 \left(s + 101\right)\left(s^2 + 1.98s + 160.6\right)}
         {\left(s - 3.8\right)\left(s + 95.3\right)\left(s^2 + 165.8s + 8643\right)}
    \label{eq:U_theta2}
\end{align}

Where we scaled the B matrix by k kw = 17.4, Voltage
to Torque, so instead of the book given derived equation where we have Torque to position, here we have voltage to position.

\subsubsection{Controller Type Analysis}
Figure \ref{fig:comp_bode} displays the frequency response of the derived compensator.

\begin{figure}[htbp]
    \centering
    \includegraphics[width=0.9\columnwidth]{Figure Part 9-11/Figures/Compensator Bode Plot.png}
    \caption{Bode plot of the combined Observer-Controller compensator.}
    \label{fig:comp_bode}
\end{figure}

The Bode plot reveals that the state-space design automatically synthesized a sophisticated controller structure equivalent to a **Notch Filter cascaded with a Lead-Lag Compensator**:

\begin{itemize}
    \item \textbf{Notch Filter (Resonance Suppression):} 
    Looking at the response from the Load ($\theta_2$) in the top-right subplot of Figure \ref{fig:comp_bode}, a distinct attenuation ("notch") is visible. This corresponds mathematically to the complex zeros in Equation (\ref{eq:U_theta2}):
    \begin{equation}
        \omega_{\text{notch}} = \sqrt{160.6} \approx 12.67 \text{ rad/s}
    \end{equation}
    This frequency aligns with the plant's resonant mode ($\omega_{res} \approx 12$ rad/s). The modern controller effectively "notches out" this frequency to prevent exciting the flexible shaft.
    
    \item \textbf{Phase Lead (Damping):} 
    The phase plot exhibits a significant phase lead near the crossover frequency. This mimics the derivative ("D") action in classical control, providing the necessary damping ($\zeta \approx 0.69$) to stabilize the oscillatory plant.
    
    \item \textbf{Integral Action:} 
    The high gain at low frequencies acts similarly to an integrator, ensuring zero steady-state error for step inputs.
\end{itemize}






% ========================================================================
% Guidelines 13 & 14: Comparative Study
% ========================================================================
\section{Comparative Study: Classical vs. Modern Control}

After spending considerable time designing and testing both control approaches on the same plant, this section summarizes what we learned about the strengths and weaknesses of each method. Both the PID controller (Guidelines 6--8) and the state-feedback controller (Guidelines 9--12) were designed for the same specs: 5\% overshoot and 2-second settling time.

\subsection{Guideline 13: Advantages and Disadvantages}

\subsubsection{Classical Control (PID)}

The proportional–integral–derivative (PID) controller was first applied due to its simplicity and widespread use in industrial control. The control law is given by
\begin{equation}
u(t) = K_p e(t) + K_i \int_0^t e(\tau)\,d\tau + K_d \dot{e}(t),
\end{equation}
where $e(t)$ denotes the tracking error.

Although the PID controller successfully stabilized the system and eliminated steady-state error for step inputs, its performance was limited by the dynamics of the plant. The torsional system is fourth-order and contains lightly damped resonant modes near $\pm16.1j$ rad/s. Root locus analysis showed that increasing the controller gains to improve transient response caused the closed-loop poles associated with these modes to move toward the right-half plane, leading to excessive oscillations or instability.

\textbf{Observed advantages of PID control include:}
\begin{itemize}
    \item Simple structure with only three tuning parameters.
    \item No explicit requirement for a detailed mathematical model.
    \item Elimination of steady-state error through integral action.
    \item Use of a single measured output signal.
\end{itemize}

\textbf{Observed limitations of PID control include:}
\begin{itemize}
    \item Inability to sufficiently damp the resonant modes without sacrificing stability.
    \item Sensitivity to measurement noise due to the derivative term.
    \item Insufficient degrees of freedom to independently shape all closed-loop poles of a fourth-order plant.
    \item Lack of a systematic mechanism for direct pole placement.
\end{itemize}

\subsubsection{Modern Control (State Feedback)}

A state-feedback controller was designed using the full state-space model of the system. The control law is
\begin{equation}
u(t) = -K\mathbf{x}(t) + \bar{N}r(t),
\end{equation}
where $K$ is the state-feedback gain matrix and $\bar{N}$ is a feedforward gain used to ensure unity steady-state tracking.

Because the system is fully controllable, the closed-loop poles could be explicitly assigned using pole-placement techniques. This allowed the lightly damped resonant modes to be relocated to well-damped positions in the left-half plane, resulting in significantly improved transient performance.

\textbf{Observed advantages of state-feedback control include:}
\begin{itemize}
    \item Explicit and guaranteed closed-loop pole placement.
    \item Effective damping of resonant modes.
    \item Systematic and repeatable design procedure.
    \item Improved robustness to measurement noise when combined with an observer.
\end{itemize}

\textbf{Observed limitations of state-feedback control include:}
\begin{itemize}
    \item Dependence on an accurate state-space model.
    \item Requirement for state estimation when not all states are measurable.
    \item Increased implementation complexity due to the controller–observer structure.
\end{itemize}

\subsection{Guideline 14: Numerical Comparison Through Simulation}

Both controllers were evaluated under identical simulation conditions in Simulink to enable a fair comparison. The key performance metrics are summarized in Table~\ref{tab:final_comparison}. This section validates the qualitative observations from Guideline 13 through side-by-side simulation results under identical plant, input, and disturbance conditions.


\begin{table}[H]
    \centering
    \caption{Side-by-Side Performance Comparison}
    \label{tab:final_comparison}
    \begin{tabular}{l c c c}
        \toprule
        \textbf{Metric} & \textbf{Target} & \textbf{PID Result} & \textbf{State FB Result} \\
        \midrule
        Percent Overshoot & $\leq 5\%$ & $\sim$20\% & $\sim$7\% \\
        Settling Time ($T_s$) & $\leq 2$ s & $\sim$15--20 s & $\sim$1.9 s \\
        Steady-State Error & 0 & 0 & 0 \\
        Small Noise Response & Stable & Stable & Stable \\
        Large Noise Response & Stable & \textbf{Failed} & Stable-ish \\
        \midrule
        \textbf{Specs Met?} & --- & \textbf{No} & \textbf{Yes} \\
        \bottomrule
    \end{tabular}
\end{table}

\subsubsection{Step Response Comparison}

The difference is stark. Figure \ref{fig:compare_step} shows both step responses side-by-side for direct comparison.

\begin{figure}[H]
    \centering
    \begin{minipage}{0.48\columnwidth}
        \centering
        \includegraphics[width=\textwidth]{Q4-8_Figures/Q7_1_PID_Step_Response.png}
        \footnotesize\textbf{(a) PID Controller}
    \end{minipage}
    \hfill
    \begin{minipage}{0.48\columnwidth}
        \centering
        \includegraphics[width=\textwidth]{Figure Part 9-11/Figures/Part9_step_input.png}
        \footnotesize\textbf{(b) State Feedback}
    \end{minipage}
    \caption{Step response comparison: (a) PID controller with $T_s \approx 15$s and $\sim$20\% overshoot; (b) State-feedback controller with $T_s \approx 1.9$s and $\sim$7\% overshoot. The state-feedback design meets the 2-second settling time specification while PID does not.}
    \label{fig:compare_step}
\end{figure}


These results indicate that the PID controller’s performance limitations are structural rather than a consequence of poor tuning. With only three adjustable parameters, it is not possible to independently assign all four closed-loop poles of the plant.

\subsubsection{Tracking Performance}

For the square wave and sinusoidal inputs, the bandwidth limitation of PID became obvious. Figure \ref{fig:compare_square} compares the square wave tracking, and Figure \ref{fig:compare_sine} compares the sinusoidal tracking.

\begin{figure}[H]
    \centering
    \begin{minipage}{0.48\columnwidth}
        \centering
        \includegraphics[width=\textwidth]{Q4-8_Figures/Q7_2_PID_Square_Response.png}
        \footnotesize\textbf{(a) PID Controller}
    \end{minipage}
    \hfill
    \begin{minipage}{0.48\columnwidth}
        \centering
        \includegraphics[width=\textwidth]{Figure Part 9-11/Figures/Part10_squareWave_input.png}
        \footnotesize\textbf{(b) State Feedback}
    \end{minipage}
    \caption{Square wave tracking comparison: (a) PID controller smooths out the sharp transitions due to limited bandwidth---output never reaches the commanded levels; (b) State-feedback tracks the square wave much more accurately with faster rise/fall times.}
    \label{fig:compare_square}
\end{figure}

\begin{figure}[H]
    \centering
    \begin{minipage}{0.48\columnwidth}
        \centering
        \includegraphics[width=\textwidth]{Q4-8_Figures/Q7_3_PID_Sine_Response.png}
        \footnotesize\textbf{(a) PID Controller}
    \end{minipage}
    \hfill
    \begin{minipage}{0.48\columnwidth}
        \centering
        \includegraphics[width=\textwidth]{Figure Part 9-11/Figures/Part10_SineWave_input_smaller_init_cond.png}
        \footnotesize\textbf{(b) State Feedback}
    \end{minipage}
    \caption{Sinusoidal tracking comparison: (a) PID controller shows significant amplitude attenuation ($\sim$70\%) and phase lag at 1 rad/s; (b) State-feedback tracks with near-unity gain and minimal phase lag due to higher closed-loop bandwidth.}
    \label{fig:compare_sine}
\end{figure}

The PID controller demonstrated limited bandwidth, resulting in smoothed square-wave responses and significant attenuation and phase lag when tracking sinusoidal inputs. In contrast, the state-feedback controller achieved improved tracking accuracy due to its higher closed-loop bandwidth and better damping of the resonant modes.

\subsubsection{Noise Robustness}
 With small noise (power $= 5 \times 10^{-7}$), both controllers worked fine. But with larger noise (power $= 0.001$), the difference was dramatic, as shown in Figure \ref{fig:compare_noise}.

\begin{figure}[H]
    \centering
    \begin{minipage}{0.48\columnwidth}
        \centering
        \includegraphics[width=\textwidth]{Q4-8_Figures/Q8_2_PID_LargeNoise_Combined.png}
        \footnotesize\textbf{(a) PID with Large Noise}
    \end{minipage}
    \hfill
    \begin{minipage}{0.48\columnwidth}
        \centering
        \includegraphics[width=\textwidth]{Figure Part 9-11/np_full_order_step_response_with_noise.png}
        \footnotesize\textbf{(b) State FB with Noise}
    \end{minipage}
    \caption{Noise robustness comparison: (a) PID controller fails with large measurement noise---the derivative term amplifies high-frequency content, causing erratic output; (b) State-feedback with full-order observer successfully filters noise through the model-based estimation, maintaining clean tracking.}
    \label{fig:compare_noise}
\end{figure}

Robustness to measurement noise was assessed by injecting band-limited white noise into the feedback signal. For low noise levels, both controllers maintained acceptable performance. However, under higher noise conditions, the PID controller exhibited significant degradation due to amplification of high-frequency noise by the derivative term.

When combined with a state observer, the state-feedback controller demonstrated superior noise rejection. The observer effectively filtered high-frequency noise by relying on model-based state prediction and correction, resulting in stable and accurate tracking even under elevated noise levels.

\subsection{Discussion: Degrees of Freedom and Controller Structure}

The observed limitations of PID control can be explained by a degrees-of-freedom argument. A PID controller introduces one pole at the origin and two zeros, leading to a fifth-order closed-loop characteristic equation:
\begin{equation}
1 + \frac{K_d s^2 + K_p s + K_i}{s} \cdot \frac{N(s)}{D(s)} = 0.
\end{equation}
This equation contains five poles but only three tunable parameters, making independent pole placement impossible.

State feedback, on the other hand, directly modifies the system dynamics:
\begin{equation}
\dot{\mathbf{x}} = (A - BK)\mathbf{x}.
\end{equation}
With four adjustable gains in $K$, the four poles of the fourth-order plant can be independently assigned, enabling precise control over transient behavior.

\subsection{Summary and Design Implications}

The comparative study shows that while PID control is capable of stabilizing the torsional system and eliminating steady-state error, it is fundamentally limited in its ability to meet stringent transient performance requirements for a fourth-order plant with lightly damped resonant modes. In contrast, state-feedback control augmented with an observer provides a systematic framework for explicitly shaping closed-loop dynamics, resulting in improved transient performance, enhanced noise robustness, and predictable stability margins. These results highlight the advantages of modern state-space methods for higher-order mechanical systems with flexible dynamics.


\section{Laboratory Work}

In addition to the simulation-based analysis, laboratory sessions were conducted using the physical ECP Model~205 torsional apparatus. Figure~\ref{fig:lab_equipment} shows the experimental setup used during open-loop testing with the ECP control software interface.

\begin{figure}[H]
    \centering
    \includegraphics[width=0.9\columnwidth]{Other_figures/Lab_ECP_Equipment.jpg}
    \caption{Team conducting open-loop response tests on the physical ECP Model~205 system using the ECP software interface.}
    \label{fig:lab_equipment}
\end{figure}

Open-loop step-response behavior was successfully observed on the physical system. The measured oscillation frequency and qualitative damping characteristics were consistent with the simulation results, providing confidence in the accuracy of the derived plant model and parameter values.


\subsection{Experimental Limitations and Apparatus Issues}

Although the open-loop behavior of the system could be observed, full experimental validation of the closed-loop controllers was not possible due to hardware limitations beyond our control.

During attempts to implement closed-loop PID control using the ECP software, the mechanical system exhibited limited physical motion; however, the corresponding sensor outputs were absent or inconsistent. As a result, no usable measurement data were recorded, and the data acquisition plots remained empty. Multiple controller configurations and gain values were tested, and troubleshooting was performed in consultation with three different teaching assistants. Following these checks, it was concluded that one or more feedback sensors (encoders) were malfunctioning, preventing reliable closed-loop operation.

Additional attempts were made to access the laboratory during the final scheduled session and on subsequent days in order to repeat the experiments or use an alternative apparatus. However, laboratory access was not available, and the hardware issue could not be resolved within the course timeframe.

Consequently, all closed-loop controller evaluations presented in this report are based on MATLAB/Simulink simulations using the verified plant model. The simulations were internally consistent across open-loop, PID-controlled, and state-feedback-controlled configurations and are sufficient to demonstrate the system dynamics, control design methodology, and comparative performance trends required by the course learning objectives. While experimental data would be necessary for full real-world validation, the inability to obtain such data was due to confirmed equipment failure rather than shortcomings in the modeling or controller design process.




\section{Conclusion}
% TODO: Final summary.

This laboratory presented a comprehensive study of classical and modern control techniques applied to a two-degree-of-freedom torsional system with lightly damped resonant modes. Starting from a transfer-function description, an equivalent state-space realization was derived and used to analyze the system’s open-loop dynamics, controllability, and observability.

A PID controller was first implemented due to its simplicity and minimal modeling requirements. While the controller successfully stabilized the system and eliminated steady-state error, its performance was fundamentally limited by the fourth-order dynamics of the plant. In particular, the presence of lightly damped resonant modes constrained achievable settling time and resulted in sensitivity to measurement noise.

To address these limitations, a full state-feedback controller augmented with a Luenberger observer was designed. By explicitly assigning the closed-loop poles, the resonant modes were effectively damped, leading to significantly improved transient response and tracking performance. The observer-based implementation also demonstrated superior robustness to measurement noise by filtering high-frequency disturbances before they entered the control loop.

The comparative results confirm that although PID control remains a practical solution for low-order systems, modern state-space methods provide clear advantages for higher-order systems with complex dynamics. The laboratory illustrates the importance of model-based design, pole placement, and observer theory in achieving reliable and high-performance control.

\vspace{10pt}\hrule\vspace{10pt}


% --- REFERENCES ---
\begin{thebibliography}{9}
    \bibitem{b1}
    ECP Model 205 Lab Manual, Chapter 5: Plant Dynamic Models.
    
    \bibitem{b2}
    ELEC 481 Linear Systems Lecture Notes, 2025.
    
    \bibitem{b3}
    Ogata, K. (2010). \textit{Modern Control Engineering} (5th ed.). Prentice Hall.

\end{thebibliography}

\newpage

% --- APPENDIX ---
\onecolumn
\appendix

% ========================================================================
% Appendix A: Guidelines 4-8 MATLAB Code
% ========================================================================
\section{Guideline 4 to 8 MATLAB Scripts}

\subsection{System Parameters and Workspace Setup (sys\_params.m)}
This script initializes all system parameters, derives the state-space model, computes the transfer function using \texttt{ss2tf()}, and sets up PID gains for Simulink.

\begin{footnotesize}
\begin{verbatim}
%% ELEC 481 - Q4-Q8 Parameter Setup for Simulink
clear; clc; close all;

%% Hardware Gain
kc = 10/32768;
kaktkp = 0.70;
ke = 16000/(2*pi);
ka = 32;
khw = kc * kaktkp * ke * ka;

%% Plant #2 Parameters
J1 = 0.0108;    % Disk 1 inertia (kg-m^2)
J2 = 0.0103;    % Disk 2 inertia (kg-m^2)
c1 = 0.007;     % Disk 1 damping (N-m-s/rad)
c2 = 0.001;     % Disk 2 damping (N-m-s/rad)
k1 = 1.37;      % Shaft stiffness (N-m/rad)

%% State-Space Matrices (x = [theta1; omega1; theta2; omega2])
A = [0,        1,         0,        0;
     -k1/J1,   -c1/J1,    k1/J1,    0;
     0,        0,         0,        1;
     k1/J2,    0,         -k1/J2,   -c2/J2];

B = [0; khw/J1; 0; 0];
C = [1, 0, 0, 0;    % theta1
     0, 0, 1, 0];   % theta2
D = [0; 0];

%% Transfer Functions (using ss2tf)
sys_ss = ss(A, B, C, D);
[num_tf, den_tf] = ss2tf(A, B, C, D);
G2 = tf(num_tf(2,:), den_tf);  % theta2/u

%% Open-Loop Pole Analysis
poles_ol = pole(G2);
fprintf('Open-Loop Poles:\n');
for i = 1:length(poles_ol)
    fprintf('  p%d = %.4f %+.4fj\n', i, real(poles_ol(i)), imag(poles_ol(i)));
end

%% Design Specs
zeta_desired = 0.69;   % For 5% overshoot
wn_desired = 3.33;     % For 2s settling time

%% Initial Conditions
X0_zero = [0; 0; 0; 0];
X0_nonzero = [0.1; 0; 0; 0];

%% PID Gains (conservative tuning)
Kp = 0.00028;
Ki = 0.000003;
Kd = 0.0001;

save('Q4_to_Q8_Workspace.mat');
\end{verbatim}
\end{footnotesize}

\subsection{Q4: Open-Loop Figure Generation (Q4\_OpenLoop\_Figures.m)}
This script generates impulse and step response plots for the open-loop system.

\begin{footnotesize}
\begin{verbatim}
%% Q4: Open-Loop System Figure Generation
clc;
fprintf('Q4: Open-Loop System Figures\n');

output_folder = 'Figures';
if ~exist(output_folder, 'dir'), mkdir(output_folder); end

%% Figure Q4.1: Impulse Response (Zero IC)
t = out.impulse_response_zero_ic.Time;
y = out.impulse_response_zero_ic.Data;

figure('Color', 'w', 'Position', [100, 100, 800, 500]);
hold on;
plot(t, y(:,1), 'k--', 'LineWidth', 1.5, 'DisplayName', 'Cmd Position');
plot(t, y(:,2), 'b-', 'LineWidth', 1.5, 'DisplayName', 'Disk 1');
plot(t, y(:,3), 'r-.', 'LineWidth', 2, 'DisplayName', 'Disk 2');
grid on; xlabel('Time (s)'); ylabel('Position (rad)');
title('Open-Loop Impulse Response (Zero IC)');
legend('show', 'Location', 'best');
saveas(gcf, fullfile(output_folder, 'Q4_1_Impulse_Zero_IC.png'));

% Similar code for Q4.2, Q4.3, Q4.4 (Non-Zero IC, Step responses)
\end{verbatim}
\end{footnotesize}

\subsection{Q5: Bode and Root Locus (Q5\_Bode\_RootLocus\_Figures.m)}
This script generates frequency-domain analysis plots.

\begin{footnotesize}
\begin{verbatim}
%% Q5: Bode Plot and Root Locus
clc;
if ~exist('G2', 'var')
    load('Q4_to_Q8_Workspace.mat');
end

output_folder = 'Figures';
zeta_desired = 0.69;
wn_desired = 3.33;

%% System Analysis
[poles_sys, zeros_sys] = pzmap(G2);
[Gm, Pm, Wcg, Wcp] = margin(G2);
fprintf('Gain Margin: %.2f dB at %.2f rad/s\n', 20*log10(Gm), Wcg);
fprintf('Phase Margin: %.2f deg at %.2f rad/s\n', Pm, Wcp);

%% Bode Plot
figure('Color', 'w');
[mag, phase, w] = bode(G2);
subplot(2,1,1);
semilogx(w, 20*log10(squeeze(mag)), 'b-', 'LineWidth', 2);
ylabel('Magnitude (dB)'); grid on;
subplot(2,1,2);
semilogx(w, squeeze(phase), 'b-', 'LineWidth', 2);
xlabel('Frequency (rad/s)'); ylabel('Phase (deg)'); grid on;
saveas(gcf, fullfile(output_folder, 'Q5_1_Bode_Uncompensated.png'));

%% Root Locus
figure('Color', 'w');
[r, k] = rlocus(G2);
hold on;
for i = 1:size(r,1)
    plot(real(r(i,:)), imag(r(i,:)), '-', 'LineWidth', 2);
end
sgrid(zeta_desired, wn_desired);
plot(real(poles_sys), imag(poles_sys), 'rx', 'MarkerSize', 14);
xlabel('Real Axis'); ylabel('Imaginary Axis');
title('Root Locus: G_2(s)');
saveas(gcf, fullfile(output_folder, 'Q5_2_RootLocus_Uncompensated.png'));
\end{verbatim}
\end{footnotesize}

\subsection{Q7-Q8: PID Response and Noise Testing (Q7\_Q8\_PID\_Figures.m)}
This script generates PID closed-loop response and robustness test plots.

\begin{footnotesize}
\begin{verbatim}
%% Q7 & Q8: PID Controller Figure Generation
clc;
output_folder = 'Figures';

% Load PID gains
if ~exist('Kp', 'var')
    load('Q4_to_Q8_Workspace.mat', 'Kp', 'Ki', 'Kd');
end
pid_gains_str = sprintf('Kp=%.2e, Ki=%.2e, Kd=%.2e', Kp, Ki, Kd);

%% Q7.1: PID Step Response
t = out.pid_step_response.Time;
y = out.pid_step_response.Data;
figure('Color', 'w');
plot(t, y(:,1), 'k--', t, y(:,2), 'b-', t, y(:,3), 'r-.', 'LineWidth', 1.5);
xlabel('Time (s)'); ylabel('Position (rad)');
title({'PID Step Response'; pid_gains_str});
legend('Reference', 'Disk 1', 'Disk 2');
saveas(gcf, fullfile(output_folder, 'Q7_1_PID_Step_Response.png'));

%% Q8: Noise Robustness Test
% Small noise (power = 5e-7)
t_resp = out.pid_small_noise_response.Time;
y_resp = out.pid_small_noise_response.Data;
t_fb = out.pid_small_noise_feedback.Time;
y_fb = out.pid_small_noise_feedback.Data;

figure('Color', 'w');
subplot(2,1,1);
plot(t_resp, y_resp(:,1), 'k--', t_fb, y_fb(:,1), 'm-', 'LineWidth', 1.5);
title('(A) Noisy Feedback'); ylabel('Position (rad)');
subplot(2,1,2);
plot(t_resp, y_resp(:,1), 'k--', t_resp, y_resp(:,3), 'b-', 'LineWidth', 1.5);
title('(B) Actual Response'); xlabel('Time (s)'); ylabel('Position (rad)');
sgtitle('Small Noise Test (Power = 5e-7)');
saveas(gcf, fullfile(output_folder, 'Q8_1_PID_SmallNoise_Combined.png'));
\end{verbatim}
\end{footnotesize}

\begin{figure}[htbp]
    \centering
    \includegraphics[width=0.7\columnwidth]{Q4-8_Figures/Appendix_Q4_Simulink_Simple.png}
    \caption{Simplified Simulink model for Q4 open-loop step response testing with zero initial conditions.}
    \label{fig:appendix_q4_simple}
\end{figure}

% ========================================================================
% Appendix B: Guidelines 9-12 MATLAB Code
% ========================================================================
\section{Guideline 9 to 12 MATLAB Scripts}
Below is the MATLAB script used to derive the controller gain ($K$), observer gain ($L$), and compensator transfer function.

\begin{footnotesize}
\begin{verbatim}
%% ELEC 481 - FINAL State-Feedback & Observer Design Script
% author : Mathias Desrochers 40241734
%
% This script designs for Modern controller Section  (Guidelines 9-12):
%   1. The State-Feedback Controller (K)
%   2. The State-Observer (L)
%   2. Full and reduced order
%   3. The Feedforward Gain (Nbar)
%   4. The matrices for the Simulink Observer Block (A_obs, B_obs...)
%
%  Must be Ran Before the simulink to initiat all fo teh variable 
clc;
clear;
close all;

%% ========================================================================
%  PHASE 1: Ini9t variable 
%  ========================================================================
fprintf('PHASE 1: given parameters\n');
%given Parameters
J1 = 0.0108;    % kg-m^2 (Inertia of Disk 1 + motor)
J2 = 0.0103;    % kg-m^2 (Inertia of Disk 2)
k1 = 1.37;      % N-m/rad (Spring constant between J1 and J2)
k2 = 0;         % N-m/rad (Spring constant of J2 to frame, 0 for free-clamped)
c1 = 0.007;     % N-m-s/rad (Damping coefficient at J1)
c2 = 0.001;     % N-m-s/rad (Damping coefficient at J2)

% --- State-Space Matrices (from manual 2dofsection, Eq 5.1-3) ---
% State vector x = [theta1; theta1_dot; theta2; theta2_dot]
A = [0,             1,            0,                0;
     -k1/J1,        -c1/J1,       k1/J1,            0;
     0,             0,            0,                1;
     k1/J2,         0,            -(k1+k2)/J2,      -c2/J2];

B = [0;
     1/J1;
     0;
     0];

% We can measure theta1 and theta2 with the encoders.
% y = [theta1; theta2]
C = [1, 0, 0, 0;
     0, 0, 1, 0];

D = [0;
     0];

% Check Controllability & Observability
ctrb_rank = rank(ctrb(A, B));
obsv_rank = rank(obsv(A, C));
fprintf('Controllability Rank: %d (Must be 4)\n', ctrb_rank);
fprintf('Observability Rank: %d (Must be 4)\n', obsv_rank);
if ctrb_rank < 4 || obsv_rank < 4
    error('System is not controllable or observable. Check parameters.');
end

%% ========================================================================
%  PHASE 2: CONTROLLER DESIGN (Guideline 9)
%  ========================================================================
fprintf('Designing Controller K...\n');

zeta = 0.69;      % ~5% Overshoot
sigma = 2.3;    %  Ts = 4.6/sigma = 2 sec
wn = sigma / zeta; % Natural frequency
wd = wn * sqrt(1-zeta^2); % Damped natural frequency

p1 = -sigma + 1j*wd;
p2 = -sigma - 1j*wd;
p3 = -sigma * 10;
p4 = -sigma * 10 - 1; 

P_controller = [p1, p2, p3, p4];

% --- Calculate Controller Gain K ---
K = place(A, B, P_controller);
fprintf('Controller Gain K:\n');
disp(K);

%% ========================================================================
%  PHASE 3: OBSERVER DESIGN (Guideline 11)
%  ========================================================================
fprintf('Designing Observer L...\n');
% --- Observer Poles (4x faster than controller) ---
observer_speed_factor = 4; 
P_observer = P_controller * observer_speed_factor;

% --- Calculate Observer Gain L ---
L = place(A', C', P_observer)';
fprintf('Observer Gain L:\n');
disp(L);

%% ========================================================================
%  PHASE 3.5: REDUCED-ORDER OBSERVER (SIMULINK MATRICES)
%  ========================================================================

% first we have to partition the original matrices (Measured 'a' vs Unmeasured 'b')
% x_a = [theta1; theta2] (Measured)
% x_b = [theta1_dot; theta2_dot] (Estimated)
Aaa = A([1,3], [1,3]);
Aab = A([1,3], [2,4]);
Aba = A([2,4], [1,3]);
Abb = A([2,4], [2,4]);

Ba = B([1,3]);
Bb = B([2,4]);

% now calculate Reduced Gain L_red
% we place 2 poles (since we only estimate 2 states)
P_red = P_observer(1:2); 
L_red = place(Abb', Aab', P_red)'; 

% 3. Calculate the Internal Matrices (Dynamics of z)
% z_dot = Ar_obs * z + Br_obs * y + Gr_obs * u
Ar_obs = Abb - L_red * Aab;
Br_obs = Ar_obs * L_red + Aba - L_red * Aaa;
Gr_obs = Bb - L_red * Ba;

% we need to resize it for simulink

% A_sim: The dynamics of the observer state 'z' (2x2)
Ar_sim = Ar_obs;
 
% [u; theta1; theta2], so we concat Gr and Br.
Br_sim = [Gr_obs, Br_obs]; 

% C_sim & D_sim: Reconstructing the full state x_hat from z and y.
% x_hat_a = y
% x_hat_b = z + L_red * y

% We construct rows to match the order: [th1, th1_dot, th2, th2_dot]

% row 1 (Theta1): Just passes through measurement y1
C1 = [0, 0];                % No z dependence
D1 = [0, 1, 0];             % Selects y1 from input [u, y1, y2]

% row 2 z1 + L_red terms
C2 = [1, 0];                % Selects z1
D2 = [0, L_red(1,1), L_red(1,2)]; % Adds L_red * y

% Row 3 (Theta2): Just passes through measurement y2
C3 = [0, 0];                % No z dependence
D3 = [0, 0, 1];             % Selects y2 from input [u, y1, y2]

% Row 4 z2 + L_red terms
C4 = [0, 1];                % Selects z2
D4 = [0, L_red(2,1), L_red(2,2)]; % Adds L_red * y

% Combine them
Cr_sim = [C1; C2; C3; C4];
Dr_sim = [D1; D2; D3; D4];

fprintf('Reduced Order Matrices done: Ar_sim, Br_sim, Cr_sim, Dr_sim\n');



%% ========================================================================
%  PHASE 4: FEEDFORWARD GAIN (Guideline 10)
%  ========================================================================
fprintf('Calculating Feedforward Gain Nbar...\n');
C_theta2 = C(2, :); % C_theta2 = [0 0 1 0]
D_theta2 = D(2);   % D_theta2 = 0


sys_cl = ss(A - B*K, B, C_theta2, D_theta2); % closed-loop system dynamics ($A-BK$) 
gain = dcgain(sys_cl); %Calculates the steady-state value (DC gain)


Nbar = 1 / gain;
fprintf('Feedforward Gain Nbar:\n');
disp(Nbar);

%% ========================================================================
%  PHASE 5: the oberserver
%  ========================================================================
%  matrices for Observer block in Simulink.
%  observer's state equation is:
% x_hat_dot = (A-LC)x_hat + [B, L] * [u; y]
%
fprintf(' Observer Matrices for Simulink...\n');

A_obs = A - L*C;

% The observer block has 2 inputs:
% 1. The control signal 'u'
% 2. The measurement 'y'  theta1  theta2
% So, B_obs combines the B and L matrices.
B_obs = [B, L]; % This will be [4x3]

% The observer's output is the full estimated state x_hat
C_obs = eye(4); % This is a [4x4] identity matrix

% The D matrix for the observer
% It has 4 outputs (C_obs) and 3 inputs (B_obs)
D_obs = zeros(4, 3);

C_system = eye(4);
D_system = zeros(4,1);


x0_arbitrary = [0.5; 0; -0.5; 0];

fprintf('\n-done ---');
fprintf('\nAll variables (A, B, C, D, K, L, Nbar, A_obs, B_obs, C_obs, D_obs) are in the workspace.\n');

%% ========================================================================
%  PHASE 6: COMPENSATOR ANALYSIS (Guideline 12)
%  ========================================================================
fprintf('\nCompensator Transfer Function...\n');

%  Compensator State-Space Matrices
A_comp = A - B*K - L*C;
B_comp = L;              
C_comp = -K;             
D_comp = zeros(1, 2);    

% Create State-Space Object
Compensator_Sys = ss(A_comp, B_comp, C_comp, D_comp);

% Add Labels
Compensator_Sys.InputName = {'\theta_1 (Motor)', '\theta_2 (Load)'};
Compensator_Sys.OutputName = {'Voltage (u)'};

% Convert to Transfer Function 
[num, den] = tfdata(Compensator_Sys, 'v');


fprintf('Compensator Transfer Functions (u / y):\n');
display(zpk(Compensator_Sys));

%Generate Bode plot
figure('Color', 'w', 'Name', 'Compensator Bode Plot');
h = bodeplot(Compensator_Sys);

% Plot Appearance
grid on;
title('Frequency Response of the Designed Compensator (Guideline 12)', 'FontSize', 12);

% Use plotting options to make it cleaner
opts = getoptions(h);
opts.Title.FontSize = 12;
opts.XLabel.FontSize = 11;
opts.YLabel.FontSize = 11;
setoptions(h, opts);

fprintf('done.\n');

\end{verbatim}
\end{footnotesize}

\begin{figure}[htbp]
    \centering
    \includegraphics[width=0.9\columnwidth]{Figure Part 9-11/Simulink 9-12/part10_simulink_diagrahm.png}
    \caption{Simulink model used for state-feedback simulation.}
    \label{fig:simulink_sf}
\end{figure}

\begin{figure}[htbp]
    \centering
    \includegraphics[width=0.9\columnwidth]{Figure Part 9-11/Simulink 9-12/Part_11_simulink_diagrahm.png}
    \caption{Simulink model including the Plant and the Full-Order Observer.}
    \label{fig:simulink_obs}
\end{figure}

\begin{figure}[htbp]
    \centering
    \includegraphics[width=0.9\columnwidth]{Figure Part 9-11/Simulink_noise_diagram.png}
    \caption{Simulink setup for noise injection testing.}
    \label{fig:noise_diagram}
\end{figure}

\end{document}